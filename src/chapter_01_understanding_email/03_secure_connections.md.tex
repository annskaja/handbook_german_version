\section{Secure Connections}

\subsection{Can other people read along when I check my email?}

As discussed in the Chapter \textbf{Basic Tips}, whether you use webmail
or an email program you should always be sure to use encryption for the
entire session, from login to logout. This will keep anyone from spying
on your communication with your email provider. Thankfully, this is
easily done due to the popular use of \emph{TLS/SSL} connections on
email servers (See appendix \textbf{TLS/SSL}).

A TLS/SSL connection in the browser, when using webmail, will appear
with \verb!https! in the URL instead of the standard \verb!http!, like
so:

\verb!https://gigglemail.com!

If your webmail host does not provide a TLS/SSL service then you should
consider discontinuing use of that account; even if your emails
themselves are not especially private or important, your account can
very easily be hacked by ``sniffing'' your password! If it is not
enabled already be sure to turn it on in your account options. At the
time of writing, Google's Gmail and Hotmail / Microsoft Live both
automatically switch your browser to using a secure connection.

If you are using an email program like Thunderbird, Mail.app or Outlook,
be sure to check that you are using TLS/SSL in the options of the
program. See the chapter \textbf{Setting Up Secure Connections} in the
section \textbf{Email Security}.

\subsection{Notes}

It's important to note that the administrators at providers like Hotmail
or Google, that host, receive or forward your email, can read your email
even if you are using secure connections. It is also worth nothing that
the private keys that Certificate Authorities sell to web site owners
can sometimes end up in the hands of governments or hackers, making it
much easier for a Man In The Middle Attack on connections using TLS/SSL
(See Glossary for ``Man in the Middle Attack''). An example is here,
implicating America's NSA and several email providers:
\href{http://cryptome.info/0001/nsa-ssl-email.htm}{http://cryptome.info/0001/nsa-ssl-email.htm}

We also note here that a \emph{Virtual Private Network} also a good way
of securing your connections when sending and reading email but requires
using a VPN client on your local machine connecting to a server. See the
chapter \textbf{Virtual Private Networking} in the \textbf{Browsing}
section.
