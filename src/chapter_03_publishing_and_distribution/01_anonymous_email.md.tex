\section{Anonymous Email}

Every data packet traveling through the Internet contains information
about its sender and its recipient. This applies to email as well as any
other network communication. There are several ways to reduce
identifying information but no way to remove it completely.

\subsection{Sending From Throw-away Email Accounts}

One option is to use a throw-away email account. This is an account set
up at a service like Gmail or Hotmail, used once or twice for anonymous
exchange. When signing up for the account, you will need to provide fake
information about your name and location. After using the account for a
short amount of time, say 24 hours, you should never log in again. If
you need to communicate further, then create a new account.

It is very important to keep in mind that these services keep logs of
the IP addresses of those using them. If you are sending highly
sensitive information, you will need to combine a throw away email
account with Tor in order keep your IP address hidden.

If you are not expecting a reply, then an anonymous remailer like
AnonEmail or Silentsender may be a useful solution. A remailer is a
server that receives messages with instructions on where to send the
data and acts as a relay, forwarding it from a generic address without
revealing the identity of the original sender. This works best when
combined with an email provider like Hushmail or RiseUp who are
specially set up for secure email connections.

Both of these methods are useful, but only if you always remember that
the intermediary himself knows where the original message came from and
can read the messages as they come in. Despite their claims to protect
your identity, these services often have user agreements that indicate
their right ``to disclose to third parties certain registration data
about you'' or they are suspected to be compromised by secret services.
The only way to safely use this technique is to not trust these services
at all, and apply extra security measures: send via Tor using a
throw-away email address.

If you only need to receive email, services like Mailinator and
MintEmail give you an email address that destroys itself after a few
hours. When signing up for any account, you should provide fake
information about your name and location and protect yourself by using
Tor.

\subsection{Be Careful about what you say!}

The content of your message can give away your identity. If you mention
details about your life, your geography, social relations or personal
appearance, people may be able to determine who is sending the message.
Even word choice and style of writing can be used to guess who might be
behind anonymous emails.

You should not use the same user name for different accounts or use a
name that you are already linked to like a childhood nickname or a
favorite book character. You should never use your secret email for
normal personal communication. If someone knows your secrets, do not
communicate with that person using this email address. If your life
depends on it, change your secret email address often as well as between
providers.

Finally, once you have your whole your email set up to protect your
identity, vanity is your worst enemy. You need to avoid being distinct.
Don't try to be clever, flamboyant or unique. Even the way you break
your paragraphs is valuable data for identification, especially these
days when every school essay and blog post you have written is available
in the Internet. Powerful organizations can actually use these texts to
build up a database that can ``fingerprint'' writing.
