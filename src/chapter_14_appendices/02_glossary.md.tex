\section{Glossary}

Much of this content is based on
\href{http://en.cship.org/wiki/Special:Allpages}{http://en.cship.org/wiki/Special:Allpages}

\subsection{aggregator}

An aggregator is a service that gathers syndicated information from one
or many sites and makes it available at a different address. Sometimes
called an RSS aggregator, a feed aggregator, a feed reader, or a news
reader. (Not to be confused with a Usenet News reader.)

\subsection{anonymity}

(Not be confused with privacy, pseudonymity, security, or
confidentiality.)

Anonymity on the Internet is the ability to use services without leaving
clues to one's identity or being spied upon. The level of protection
depends on the anonymity techniques used and the extent of monitoring.
The strongest techniques in use to protect anonymity involve creating a
chain of communication using a random process to select some of the
links, in which each link has access to only partial information about
the process. The first knows the user's Internet address (IP) but not
the content, destination, or purpose of the communication, because the
message contents and destination information are encrypted. The last
knows the identity of the site being contacted, but not the source of
the session. One or more steps in between prevents the first and last
links from sharing their partial knowledge in order to connect the user
and the target site.

\subsection{anonymous remailer}

An anonymous remailer is a service that accepts e-mail messages
containing instructions for delivery, and sends them out without
revealing their sources. Since the remailer has access to the user's
address, the content of the message, and the destination of the message,
remailers should be used as part of a chain of multiple remailers so
that no one remailer knows all this information.

\subsection{ASP (application service provider)}

An ASP is an organization that offers software services over the
Internet, allowing the software to be upgraded and maintained centrally.

\subsection{backbone}

A backbone is one of the high-bandwidth communications links that tie
together networks in different countries and organizations around the
world to form the Internet.

\subsection{badware}

See malware.

\subsection{bandwidth}

The bandwidth of a connection is the maximum rate of data transfer on
that connection, limited by its capacity and the capabilities of the
computers at both ends of the connection.

\subsection{bash (Bourne-again shell)}

The bash shell is a command-line interface for Linux/Unix operating
systems, based on the Bourne shell.

\subsection{BitTorrent}

BitTorrent is a peer-to-peer file-sharing protocol invented by Bram
Cohen in 2001. It allows individuals to cheaply and effectively
distribute large files, such as CD images, video, or music files.

\subsection{blacklist}

A blacklist is a list of forbidden things. In Internet censorship, lists
of forbidden Web sites or the IP addresses of computers may be used as
blacklists; censorware may allow access to all sites except for those
specifically listed on its blacklist. An alternative to a blacklist is a
whitelist, or a list of permitted things. A whitelist system blocks
access to all sites except for those specifically listed on the
whitelist. This is a less common approach to Internet censorship. It is
possible to combine both approaches, using string matching or other
conditional techniques on URLs that do not match either list.

\subsection{bluebar}

The blue URL bar (called the Bluebar in Psiphon lingo) is the form at
the top of your Psiphon node browser window, which allows you to access
blocked site by typing its URL inside.

See also Psiphon node

\subsection{block}

To block is to prevent access to an Internet resource, using any number
of methods.

\subsection{bookmark}

A bookmark is a placeholder within software that contains a reference to
an external resource. In a browser, a bookmark is a reference to a Web
page -- by choosing the bookmark you can quickly load the Web site
without needing to type in the full URL.

\subsection{bridge}

See Tor bridge.

\subsection{brute-force attack}

A brute force attack consists of trying every possible code,
combination, or password until you find the right one. These are some of
the most trivial hacking attacks.

\subsection{cache}

A cache is a part of an information-processing system used to store
recently used or frequently used data to speed up repeated access to it.
A Web cache holds copies of Web page files.

\subsection{censor}

To censor is to prevent publication or retrieval of information, or take
action, legal or otherwise, against publishers and readers.

\subsection{censorware}

Censorware is software used to filter or block access to the Internet.
This term is most often used to refer to Internet filtering or blocking
software installed on the client machine (the PC which is used to access
the Internet). Most such client-side censorware is used for parental
control purposes.

Sometimes the term censorware is also used to refer to software used for
the same purpose installed on a network server or router.

\subsection{CGI (Common Gateway Interface)}

CGI is a common standard used to let programs on a Web server run as Web
applications. Many Web-based proxies use CGI and thus are also called
``CGI proxies''. (One popular CGI proxy application written by James
Marshall using the Perl programming language is called CGIProxy.)

\subsection{chat}

Chat, also called instant messaging, is a common method of communication
among two or more people in which each line typed by a participant in a
session is echoed to all of the others. There are numerous chat
protocols, including those created by specific companies (AOL, Yahoo!,
Microsoft, Google, and others) and publicly defined protocols. Some chat
client software uses only one of these protocols, while others use a
range of popular protocols.

\subsection{cipher}

In cryptography, a cipher (or cypher) is an algorithm for performing
encryption or decryption

\subsection{circumvention}

Circumvention is publishing or accessing content in spite of attempts at
censorship.

\subsection{Common Gateway Interface}

See CGI.

\subsection{command-line interface}

A method of controlling the execution of software using commands entered
on a keyboard, such as a Unix shell or the Windows command line.

\subsection{cookie}

A cookie is a text string sent by a Web server to the user's browser to
store on the user's computer, containing information needed to maintain
continuity in sessions across multiple Web pages, or across multiple
sessions. Some Web sites cannot be used without accepting and storing a
cookie. Some people consider this an invasion of privacy or a security
risk.

\subsection{country code top-level domain (ccTLD)}

Each country has a two-letter country code, and a TLD (top-level domain)
based on it, such as .ca for Canada; this domain is called a country
code top-level domain. Each such ccTLD has a DNS server that lists all
second-level domains within the TLD. The Internet root servers point to
all TLDs, and cache frequently-used information on lower-level domains.

\subsection{cryptography}

Cryptography is the practice and study of techniques for secure
communication in the presence of third parties (called adversaries).
More generally, it is about constructing and analyzing protocols that
overcome the influence of adversaries and which are related to various
aspects in information security such as data confidentiality, data
integrity, authentication, and non-repudiation. Modern cryptography
intersects the disciplines of mathematics, computer science, and
electrical engineering. Applications of cryptography include ATM cards,
computer passwords, and electronic commerce.

\subsection{DARPA (Defense Advanced Projects Research Agency)}

DARPA is the successor to ARPA, which funded the Internet and its
predecessor, the ARPAnet.

\subsection{decryption}

Decryption is recovering plain text or other messages from encrypted
data with the use of a key.

See also encryption.

\subsection{disk encryption}

Disk encryption is a technology which protects information by converting
it into unreadable code that cannot be deciphered easily by unauthorized
people. Disk encryption uses disk encryption software or hardware to
encrypt every bit of data that goes on a disk or disk volume. Disk
encryption prevents unauthorized access to data storage.

\subsection{domain}

A domain can be a Top-Level Domain (TLD) or secondary domain on the
Internet.

See also Top-Level Domain, country code Top-Level Domain and secondary
domain.

\subsection{DNS (Domain Name System)}

The Domain Name System (DNS) converts domain names, made up of
easy-to-remember combinations of letters, to IP addresses, which are
hard-to-remember strings of numbers. Every computer on the Internet has
a unique address (a little bit like an area code+telephone number).

\subsection{DNS leak}

A DNS leak occurs when a computer configured to use a proxy for its
Internet connection nonetheless makes DNS queries without using the
proxy, thus exposing the user's attempts to connect with blocked sites.
Some Web browsers have configuration options to force the use of the
proxy.

\subsection{DNS server}

A DNS server, or name server, is a server that provides the look-up
function of the Domain Name System. It does this either by accessing an
existing cached record of the IP address of a specific domain, or by
sending a request for information to another name server.

\subsection{DNS tunnel}

A DNS tunnel is a way to tunnel almost everything over DNS/Nameservers.

Because you ``abuse'' the DNS system for an unintended purpose, it only
allows a very slow connection of about 3 kb/s which is even less than
the speed of an analog modem. That is not enough for YouTube or file
sharing, but should be sufficient for instant messengers like ICQ or MSN
Messenger and also for plain text e-mail.

On the connection you want to use a DNS tunnel, you only need port 53 to
be open; therefore it even works on many commercial Wi-Fi providers
without the need to pay.

The main problem is that there are no public modified nameservers that
you can use. You have to set up your own. You need a server with a
permanent connection to the Internet running Linux. There you can
install the free software OzymanDNS and in combination with SSH and a
proxy like Squid you can use the tunnel. More Information on this on
http://www.dnstunnel.de. eavesdropping

Eavesdropping is listening to voice traffic or reading or filtering data
traffic on a telephone line or digital data connection, usually to
detect or prevent illegal or unwanted activities or to control or
monitor what people are talking about.

\subsection{e-mail}

E-mail, short for electronic mail, is a method to send and receive
messages over the Internet. It is possible to use a Web mail service or
to send e-mails with the SMTP protocol and receive them with the POP3
protocol by using an e-mail client such as Outlook Express or
Thunderbird. It is comparatively rare for a government to block e-mail,
but e-mail surveillance is common. If e-mail is not encrypted, it could
be read easily by a network operator or government.

\subsection{embedded script}

An embedded script is a piece of software code.

\subsection{encryption}

Encryption is any method for recoding and scrambling data or
transforming it mathematically to make it unreadable to a third party
who doesn't know the secret key to decrypt it. It is possible to encrypt
data on your local hard drive using software like TrueCrypt
(http://www.truecrypt.org) or to encrypt Internet traffic with TLS/SSL
or SSH.

See also decryption.

\subsection{exit node}

An exit node is a Tor node that forwards data outside the Tor network.

See also middleman node.

\subsection{file sharing}

File sharing refers to any computer system where multiple people can use
the same information, but often refers to making music, films or other
materials available to others free of charge over the Internet.

\subsection{file spreading engine}

A file spreading engine is a Web site a publisher can use to get around
censorship. A user only has to upload a file to publish once and the
file spreading engine uploads that file to some set of sharehosting
services (like Rapidshare or Megaupload).

\subsection{filter}

To filter is to search in various ways for specific data patterns to
block or permit communications.

\subsection{Firefox}

Firefox is the most popular free and open source Web browser, developed
by the Mozilla Foundation.

\subsection{forum}

On a Web site, a forum is a place for discussion, where users can post
messages and comment on previously posted messages. It is distinguished
from a mailing list or a Usenet newsgroup by the persistence of the
pages containing the message threads. Newsgroup and mailing list
archives, in contrast, typically display messages one per page, with
navigation pages listing only the headers of the messages in a thread.

\subsection{frame}

A frame is a portion of a Web page with its own separate URL. For
example, frames are frequently used to place a static menu next to a
scrolling text window.

\subsection{FTP (File Transfer Protocol)}

The FTP protocol is used for file transfers. Many people use it mostly
for downloads; it can also be used to upload Web pages and scripts to
some Web servers. It normally uses ports 20 and 21, which are sometimes
blocked. Some FTP servers listen to an uncommon port, which can evade
port-based blocking.

A popular free and open source FTP client for Windows and Mac OS is
FileZilla. There are also some Web-based FTP clients that you can use
with a normal Web browser like Firefox.

\subsection{full disk encryption}

see disk encryption.

\subsection{gateway}

A gateway is a node connecting two networks on the Internet. An
important example is a national gateway that requires all incoming or
outgoing traffic to go through it.

\subsection{GNU Privacy Guard}

GNU Privacy Guard (GnuPG or GPG) is a GPL Licensed alternative to the
PGP suite of cryptographic software. GnuPG is compliant with RFC 4880,
which is the current IETF standards track specification of OpenPGP.

see also Pretty Good Privacy (PGP).

\subsection{GPG}

see GNU Privacy Guard.

\subsection{honeypot}

A honeypot is a site that pretends to offer a service in order to entice
potential users to use it, and to capture information about them or
their activities.

\subsection{hop}

A hop is a link in a chain of packet transfers from one computer to
another, or any computer along the route. The number of hops between
computers can give a rough measure of the delay (latency) in
communications between them. Each individual hop is also an entity that
has the ability to eavesdrop on, block, or tamper with communications.

\subsection{HTTP (Hypertext Transfer Protocol)}

HTTP is the fundamental protocol of the World Wide Web, providing
methods for requesting and serving Web pages, querying and generating
answers to queries, and accessing a wide range of services.

\subsection{HTTPS (Secure HTTP)}

Secure HTTP is a protocol for secure communication using encrypted HTTP
messages. Messages between client and server are encrypted in both
directions, using keys generated when the connection is requested and
exchanged securely. Source and destination IP addresses are in the
headers of every packet, so HTTPS cannot hide the fact of the
communication, just the contents of the data transmitted and received.

\subsection{IANA (Internet Assigned Numbers Authority)}

IANA is the organization responsible for technical work in managing the
infrastructure of the Internet, including assigning blocks of IP
addresses for top-level domains and licensing domain registrars for
ccTLDs and for the generic TLDs, running the root name servers of the
Internet, and other duties.

\subsection{ICANN (Internet Corporation for Assigned Names and Numbers)}

ICANN is a corporation created by the US Department of Commerce to
manage the highest levels of the Internet. Its technical work is
performed by IANA.

\subsection{Instant Messaging (IM)}

Instant messaging is either certain proprietary forms of chat using
proprietary protocols, or chat in general. Common instant messaging
clients include MSN Messenger, ICQ, AIM or Yahoo! Messenger.
intermediary

See man in the middle.

\subsection{Internet}

The Internet is a network of networks interconnected using TCP/IP and
other communication protocols. IP (Internet Protocol) Address

An IP address is a number identifying a particular computer on the
Internet. In the previous version 4 of the Internet Protocol an IP
address consisted of four bytes (32 bits), often represented as four
integers in the range 0--255 separated by dots, such as 74.54.30.85. In
IPv6, which the Net is currently switching to, an IP address is four
times longer, and consists of 16 bytes (128 bits). It can be written as
8 groups of 4 hex digits separated by colons, such as
2001:0db8:85a3:0000:0000:8a2e:0370:7334.

\subsection{IRC (Internet relay chat)}

IRC is a more than 20-year-old Internet protocol used for real-time text
conversations (chat or instant messaging). There exist several IRC
networks --- the largest have more than 50 000 users.

\subsection{ISP (Internet Service Provider)}

An ISP (Internet service provider) is a business or organization that
provides access to the Internet for its customers.

\subsection{JavaScript}

JavaScript is a scripting language, commonly used in Web pages to
provide interactive functions.

\subsection{KeePass, KeePassX}

KeePass and KeePassX are types of Password Manager.

\subsection{keychain software}

see Password Manager.

\subsection{keyword filter}

A keyword filter scans all Internet traffic going through a server for
forbidden words or terms to block.

\subsection{latency}

Latency is a measure of time delay experienced in a system, here in a
computer network. It is measured by the time between the start of packet
transmission to the start of packet reception, between one network end
(e.g.~you) to the other end (e.g.~the Web server). One very powerful way
of Web filtering is maintaining a very high latency, which makes lots of
circumvention tools very difficult to use.

\subsection{log file}

A log file is a file that records a sequence of messages from a software
process, which can be an application or a component of the operating
system. For example, Web servers or proxies may keep log files
containing records about which IP addresses used these services when and
what pages were accessed.

\subsection{low-bandwidth filter}

A low-bandwidth filter is a Web service that removes extraneous elements
such as advertising and images from a Web page and otherwise compresses
it, making page download much quicker.

\subsection{malware}

Malware is a general term for malicious software, including viruses,
that may be installed or executed without your knowledge. Malware may
take control of your computer for purposes such as sending spam.
(Malware is also sometimes called badware.)

\subsection{man in the middle}

A man in the middle or man-in-the-middle is a person or computer
capturing traffic on a communication channel, especially to selectively
change or block content in a way that undermines cryptographic security.
Generally the man-in-the-middle attack involves impersonating a Web
site, service, or individual in order to record or alter communications.
Governments can run man-in-the-middle attacks at country gateways where
all traffic entering or leaving the country must pass.

\subsection{middleman node}

A middleman node is a Tor node that is not an exit node. Running a
middleman node can be safer than running an exit node because a
middleman node will not show up in third parties' log files. (A
middleman node is sometimes called a non-exit node.)

\subsection{monitor}

To monitor is to check a data stream continuously for unwanted activity.

\subsection{network address translation (NAT)}

NAT is a router function for hiding an address space by remapping. All
traffic going out from the router then uses the router's IP address, and
the router knows how to route incoming traffic to the requestor. NAT is
frequently implemented by firewalls. Because incoming connections are
normally forbidden by NAT, NAT makes it difficult to offer a service to
the general public, such as a Web site or public proxy. On a network
where NAT is in use, offering such a service requires some kind of
firewall configuration or NAT traversal method.

\subsection{network operator}

A network operator is a person or organization who runs or controls a
network and thus is in a position to monitor, block, or alter
communications passing through that network.

\subsection{node}

A node is an active device on a network. A router is an example of a
node. In the Psiphon and Tor networks, a server is referred to as a
node.

\subsection{non-exit node}

See middleman node.

\subsection{obfuscation}

Obfuscation means obscuring text using easily-understood and
easily-reversed transformation techniques that will withstand casual
inspection but not cryptanalysis, or making minor changes in text
strings to prevent simple matches. Web proxies often use obfuscation to
hide certain names and addresses from simple text filters that might be
fooled by the obfuscation. As another example, any domain name can
optionally contain a final dot, as in ``somewhere.com.'', but some
filters might search only for ``somewhere.com'' (without the final dot).

\subsection{open node}

An open node is a specific Psiphon node which can be used without
logging in. It automatically loads a particular homepage, and presents
itself in a particular language, but can then be used to browse
elsewhere.

See also Psiphon node.

\subsection{OTR/Off-the-Record messaging}

Off-the-Record Messaging, commonly referred to as OTR, is a
cryptographic protocol that provides strong encryption for instant
messaging conversations.

\subsection{packet}

A packet is a data structure defined by a communication protocol to
contain specific information in specific forms, together with arbitrary
data to be communicated from one point to another. Messages are broken
into pieces that will fit in a packet for transmission, and reassembled
at the other end of the link.

\subsection{password manager}

A password manager is software that helps a user organize passwords and
PIN codes. The software typically has a local database or a file that
holds the encrypted password data for secure logon onto computers,
networks, web sites and application data files. KeePass
http://keepass.info/ is an example of a password manager.

\subsection{pastebin}

A web service where any kind of text can be dumped and read without
registration. All text will be visible publicly.

\subsection{peer-to-peer}

A peer-to-peer (or P2P) network is a computer network between equal
peers. Unlike client-server networks there is no central server and so
the traffic is distributed only among the clients.This technology is
mostly applied to file sharing programs like BitTorrent, eMule and
Gnutella. But also the very old Usenet technology or the VoIP program
Skype can be categorized as peer-to-peer systems.

See also file sharing.

\subsection{perfect forward secrecy}

In an authenticated key-agreement protocol that uses public key
cryptography, perfect forward secrecy (or PFS) is the property that
ensures that a session key derived from a set of long-term public and
private keys will not be compromised if one of the (long-term) private
keys is compromised in the future.

\subsection{Pretty Good Privacy (PGP)}

Pretty Good Privacy (PGP) is a data encryption and decryption computer
program that provides cryptographic privacy and authentication for data
communication. PGP is often used for signing, encrypting and decrypting
texts, e-mails, files, directories and whole disk partitions to increase
the security of e-mail communications.

PGP and similar products follow the OpenPGP standard (RFC 4880) for
encrypting and decrypting data.

\subsection{PHP}

PHP is a scripting language designed to create dynamic Web sites and web
applications. It is installed on a Web server. For example, the popular
Web proxy PHProxy uses this technology.

\subsection{plain text}

Plain text is unformatted text consisting of a sequence of character
codes, as in ASCII plain text or Unicode plain text.

\subsection{plaintext}

Plaintext is unencrypted text, or decrypted text.

See also encryption, TLS/SSL, SSH.

\subsection{privacy}

Protection of personal privacy means preventing disclosure of personal
information without the permission of the person concerned. In the
context of circumvention, it means preventing observers from finding out
that a person has sought or received information that has been blocked
or is illegal in the country where that person is at the time.

\subsection{private key}

see public key encryption/public-key cryptography.

\subsection{POP3}

Post Office Protocol version 3 is used to receive mail from a server, by
default on port 110 with an e-mail program such as Outlook Express or
Thunderbird.

\subsection{port}

A hardware port on a computer is a physical connector for a specific
purpose, using a particular hardware protocol. Examples are a VGA
display port or a USB connector.

Software ports also connect computers and other devices over networks
using various protocols, but they exist in software only as numbers.
Ports are somewhat like numbered doors into different rooms, each for a
special service on a server or PC. They are identified by numbers from 0
to 65535.

\subsection{protocol}

A formal definition of a method of communication, and the form of data
to be transmitted to accomplish it. Also, the purpose of such a method
of communication. For example, Internet Protocol (IP) for transmitting
data packets on the Internet, or Hypertext Transfer Protocol for
interactions on the World Wide Web.

\subsection{proxy server}

A proxy server is a server, a computer system or an application program
which acts as a gateway between a client and a Web server. A client
connects to the proxy server to request a Web page from a different
server. Then the proxy server accesses the resource by connecting to the
specified server, and returns the information to the requesting site.
Proxy servers can serve many different purposes, including restricting
Web access or helping users route around obstacles.

\subsection{Psiphon node}

A Psiphon node is a secured web proxy designed to evade Internet
censorship. It is developed by Psiphon inc. Psiphon nodes can be open or
private.

\subsection{private node}

A private node is a Psiphon node working with authentication, which
means that you have to register before you can use it. Once registered,
you will be able to send invitations to your friends and relatives to
use this specific node.

See also Psiphon node.

\subsection{public key}

see public key encryption/public-key cryptography.

\subsection{public key encryption/public-key cryptography}

Public-key cryptography refers to a cryptographic system requiring two
separate keys, one of which is secret and one of which is public.
Although different, the two parts of the key pair are mathematically
linked. One key locks or encrypts the plaintext, and the other unlocks
or decrypts the ciphertext. Neither key can perform both functions. One
of these keys is published or public, while the other is kept private.

Public-key cryptography uses asymmetric key algorithms (such as RSA),
and can also be referred to by the more generic term ``asymmetric key
cryptography.''

\subsection{publicly routable IP address}

Publicly routable IP addresses (sometimes called public IP addresses)
are those reachable in the normal way on the Internet, through a chain
of routers. Some IP addresses are private, such as the 192.168.x.x
block, and many are unassigned.

\subsection{regular expression}

A regular expression (also called a regexp or RE) is a text pattern that
specifies a set of text strings in a particular regular expression
implementation such as the UNIX grep utility. A text string ``matches''
a regular expression if the string conforms to the pattern, as defined
by the regular expression syntax. In each RE syntax, some characters
have special meanings, to allow one pattern to match multiple other
strings. For example, the regular expression lo+se matches lose, loose,
and looose.

\subsection{remailer}

An anonymous remailer is a service which allows users to send e-mails
anonymously. The remailer receives messages via e-mail and forwards them
to their intended recipient after removing information that would
identify the original sender. Some also provide an anonymous return
address that can be used to reply to the original sender without
disclosing her identity. Well-known Remailer services include
Cypherpunk, Mixmaster and Nym.

\subsection{router}

A router is a computer that determines the route for forwarding packets.
It uses address information in the packet header and cached information
on the server to match address numbers with hardware connections.

\subsection{root name server}

A root name server or root server is any of thirteen server clusters run
by IANA to direct traffic to all of the TLDs, as the core of the DNS
system.

\subsection{RSS (Real Simple Syndication)}

RSS is a method and protocol for allowing Internet users to subscribe to
content from a Web page, and receive updates as soon as they are posted.

\subsection{scheme}

On the Web, a scheme is a mapping from a name to a protocol. Thus the
HTTP scheme maps URLs that begin with HTTP: to the Hypertext Transfer
Protocol. The protocol determines the interpretation of the rest of the
URL, so that http://www.example.com/dir/content.html identifies a Web
site and a specific file in a specific directory, and
mailto:user@somewhere.com is an e-mail address of a specific person or
group at a specific domain.

\subsection{shell}

A UNIX shell is the traditional command line user interface for the
UNIX/Linux operating systems. The most common shells are sh and bash.

\subsection{SOCKS}

A SOCKS proxy is a special kind of proxy server. In the ISO/OSI model it
operates between the application layer and the transport layer. The
standard port for SOCKS proxies is 1080, but they can also run on
different ports. Many programs support a connection through a SOCKS
proxy. If not you can install a SOCKS client like FreeCap, ProxyCap or
SocksCap which can force programs to run through the Socks proxy using
dynamic port forwarding. It is also possible to use SSH tools such as
OpenSSH as a SOCKS proxy server.

\subsection{screenlogger}

A screenlogger is software able to record everything your computer
displays on the screen. The main feature of a screenlogger is to capture
the screen and log it into files to view at any time in the future.
Screen loggers can be used as powerful monitoring tool. You should be
aware of any screen logger running on any computer you are using,
anytime.

\subsection{script}

A script is a program, usually written in an interpreted, non-compiled
language such as JavaScript, Java, or a command interpreter language
such as bash. Many Web pages include scripts to manage user interaction
with a Web page, so that the server does not have to send a new page for
each change.

\subsection{smartphone}

A smartphone is a mobile phone that offers more advanced computing
ability and connectivity than a contemporary feature phone, such as Web
access, ability to run elaborated operating systems and run built-in
applications.

\subsection{spam}

Spam is messages that overwhelm a communications channel used by people,
most notably commercial advertising sent to large numbers of individuals
or discussion groups. Most spam advertises products or services that are
illegal in one or more ways, almost always including fraud. Content
filtering of e-mail to block spam, with the permission of the recipient,
is almost universally approved of.

\subsection{SSH (Secure Shell)}

SSH or Secure Shell is a network protocol that allows encrypted
communication between computers. It was invented as a successor of the
unencrypted Telnet protocol and is also used to access a shell on a
remote server.

The standard SSH port is 22. It can be used to bypass Internet
censorship with port forwarding or it can be used to tunnel other
programs like VNC.

\subsection{SSL (Secure Sockets Layer)}

SSL (or Secure Sockets Layer), is one of several cryptographic standards
used to make Internet transactions secure. It is was used as the basis
for the creation of the related Transport Layer Security (TLS). You can
easily see if you are using SSL by looking at the URL in your Browser
(like Firefox or Internet Explorer): If it starts with https instead of
http, your connection is encrypted.

\subsection{steganography}

Steganography, from the Greek for hidden writing, refers to a variety of
methods of sending hidden messages where not only the content of the
message is hidden but the very fact that something covert is being sent
is also concealed. Usually this is done by concealing something within
something else, like a picture or a text about something innocent or
completely unrelated. Unlike cryptography, where it is clear that a
secret message is being transmitted, steganography does not attract
attention to the fact that someone is trying to conceal or encrypt a
message.

\subsection{subdomain}

A subdomain is part of a larger domain. If for example ``wikipedia.org''
is the domain for the Wikipedia, ``en.wikipedia.org'' is the subdomain
for the English version of the Wikipedia.

\subsection{threat analysis}

A security threat analysis is properly a detailed, formal study of all
known ways of attacking the security of servers or protocols, or of
methods for using them for a particular purpose such as circumvention.
Threats can be technical, such as code-breaking or exploiting software
bugs, or social, such as stealing passwords or bribing someone who has
special knowledge. Few companies or individuals have the knowledge and
skill to do a comprehensive threat analysis, but everybody involved in
circumvention has to make some estimate of the issues.

\subsection{Top-Level Domain (TLD)}

In Internet names, the TLD is the last component of the domain name.
There are several generic TLDs, most notably .com, .org, .edu, .net,
.gov, .mil, .int, and one two-letter country code (ccTLD) for each
country in the system, such as .ca for Canada. The European Union also
has the two-letter code .eu.

\subsection{TLS (Transport Layer Security)}

TLS or Transport Layer Security is a cryptographic standard based on
SSL, used to make Internet transactions secure.

\subsection{TCP/IP (Transmission Control Protocol over Internet
Protocol)}

TCP and IP are the fundamental protocols of the Internet, handling
packet transmission and routing. There are a few alternative protocols
that are used at this level of Internet structure, such as UDP.

\subsection{Tor bridge}

A bridge is a middleman Tor node that is not listed in the main public
Tor directory, and so is possibly useful in countries where the public
relays are blocked. Unlike the case of exit nodes, IP addresses of
bridge nodes never appear in server log files and never pass through
monitoring nodes in a way that can be connected with circumvention.

\subsection{traffic analysis}

Traffic analysis is statistical analysis of encrypted communications. In
some circumstances traffic analysis can reveal information about the
people communicating and the information being communicated.

\subsection{tunnel}

A tunnel is an alternate route from one computer to another, usually
including a protocol that specifies encryption of messages.

\subsection{UDP (User Datagram Packet)}

UDP is an alternate protocol used with IP. Most Internet services can be
accessed using either TCP or UDP, but there are some that are defined to
use only one of these alternatives. UDP is especially useful for
real-time multimedia applications like Internet phone calls (VoIP).

\subsection{URL (Uniform Resource Locator)}

The URL (Uniform Resource Locator) is the address of a Web site. For
example, the URL for the World News section of the NY Times is
http://www.nytimes.com/pages/world/index.html. Many censoring systems
can block a single URL. Sometimes an easy way to bypass the block is to
obscure the URL. It is for example possible to add a dot after the site
name, so the URL http://en.cship.org/wiki/URL becomes
http://en.cship.org./wiki/URL. If you are lucky with this little trick
you can access blocked Web sites.

\subsection{Usenet}

Usenet is a more than 20-year-old discussion forum system accessed using
the NNTP protocol. The messages are not stored on one server but on many
servers which distribute their content constantly. Because of that it is
impossible to censor Usenet as a whole, however access to Usenet can and
is often blocked, and any particular server is likely to carry only a
subset of locally-acceptable Usenet newsgroups. Google archives the
entire available history of Usenet messages for searching. VoIP (Voice
over Internet Protocol)

VoIP refers to any of several protocols for real-time two-way voice
communication on the Internet, which is usually much less expensive than
calling over telephone company voice networks. It is not subject to the
kinds of wiretapping practiced on telephone networks, but can be
monitored using digital technology. Many companies produce software and
equipment to eavesdrop on VoIP calls; securely encrypted VoIP
technologies have only recently begun to emerge.

\subsection{VPN (virtual private network)}

A VPN (virtual private network) is a private communication network used
by many companies and organizations to connect securely over a public
network. Usually on the Internet it is encrypted and so nobody except
the endpoints of the communication can look at the data traffic. There
are various standards like IPSec, SSL, TLS. The use of a VPN provider is
a very fast, secure and convenient method to bypass Internet censorship
with little risks but it generally costs money every month. Further,
note that the VPN standard PPTP is no longer considered secure, and
should be avoided.

\subsection{whitelist}

A whitelist is a list of sites specifically authorized for a particular
form of communication. Filtering traffic can be done either by a
whitelist (block everything but the sites on the list), a blacklist
(allow everything but the sites on the list), a combination of the two,
or by other policies based on specific rules and conditions.

\subsection{World Wide Web (WWW)}

The World Wide Web is the network of hyperlinked domains and content
pages accessible using the Hypertext Transfer Protocol and its numerous
extensions. The World Wide Web is the most famous part of the Internet.

\subsection{Webmail}

Webmail is e-mail service through a Web site. The service sends and
receives mail messages for users in the usual way, but provides a Web
interface for reading and managing messages, as an alternative to
running a mail client such as Outlook Express or Thunderbird on the
user's computer. For example a popular and free webmail service is
https://mail.google.com/

\subsection{Web proxy}

A Web proxy is a script running on a Web server which acts as a
proxy/gateway. Users can access such a Web proxy with their normal Web
browser (like Firefox) and enter any URL in the form located on that Web
site. Then the Web proxy program on the server receives that Web content
and displays it to the user. This way the ISP only sees a connection to
the server with the Web proxy since there is no direct connection.

\subsection{WHOIS}

WHOIS (who is) is the aptly named Internet function that allows one to
query remote WHOIS databases for domain registration information. By
performing a simple WHOIS search you can discover when and by whom a
domain was registered, contact information, and more.

A WHOIS search can also reveal the name or network mapped to a numerical
IP address
