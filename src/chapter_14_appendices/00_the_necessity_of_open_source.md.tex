\section{The necessity of Open Source}

The last 20 years have seen network technology reaching ever more deeply
into our lives, informing how we communicate and act within the world.
With this come inherent risks: the less we understand the network
environment we depend upon, the more vulnerable we are to exploitation.

This ignorance is something traditionally enjoyed by criminals. In
recent years however some corporations and governments have exploited
civilian ignorance in a quest for increased control. This flagrant and
often covert denial of dignity breaches many basic rights, the right to
privacy, in particular.

Closed source software has been a great boon to such exploitation --
primarily due to the fact there is no code available for open,
decentralised security auditing by the community . Under the auspices of
hiding trade secrets, closed-source software developers have proven to
be unwilling to explain to users how their programs work. This might not
always be an issue were it not for the high stakes: identity theft, the
distribution of deeply personal opinion and sentiment, a persons diverse
interests and even his/her home increasingly come into close contact
with software in a world-wide network context. As such, many people find
themselves using software for personal purposes with full trust that it
are secure. The Windows operating system itself is the most obvious
real-world example. Apple's OS X follows close behind, with large
portions of the operating system's inner-workings barred from public
inspection.

In Cryptography there is a strong principle, established in the 19th
century by \emph{Auguste Kerckhoff} (and hence named after him) which
demands that

\begin{quote}
``{[}the encryption method{]} must not be required to be secret, and it
must be able to fall into the hands of the enemy without
inconvenience''.

\end{quote}
While this principle has been taken further by most scientific and (of
course) open source communities -- publishing their methods and
inner-workings upfront, so potential weaknesses can be pointed out and
fixed before further distribution -- most distributors of proprietary
software rely on obfuscation to hide the weaknesses of their software.
As such they often prove to address newly discovered vulnerabilities in
a non-transparent way -- leaving many trusting users at risk of
exploitation.

Of course it must be said that Open Source Software is as secure as you
make it (and there is a lot of OSS written by beginners). However there
are many good examples of well written, well managed software which have
such a large (and concerned) user base that even the tiniest of mistakes
are quickly found and dealt with. This is especially the case with
software depended upon in a network context.

To use closed source software in a network context is not only to be a
minority, it is to be overlooked by a vast community of concerned
researchers and specialists that have your privacy and safety in mind.

N.B. There is also a more cynical view of Open Source Software, which
points out that since nobody is paid full time to constantly review and
regression test the latest tinkering by unskilled or deliberately
malicious programmers, it can also suffer from major security weaknesses
which go undetected for long periods of time in complicated software,
leaving it vulnerable to hackers, criminals and intelligence agencies
etc. e.g.~the (now fixed) Debian Linux predictable random number
generator problem which led to the creation of lots of weak
cryptographic keys.
