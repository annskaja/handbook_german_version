\section{Accounts and Security}

When you browse, you may be logged into various services, sometimes at
the same time. It may be a company website, your email or a social
networking site. Our accounts are important to us because highly
sensitive information about us and others is stored on machines
elsewhere on the Internet.

Keeping your accounts secure requires more than just a strong password
(see section \textbf{Passwords}) and a secure communication link with
the server via TLS/SSL (see chapter \textbf{Secure Connection}). Unless
specified otherwise, most browsers will store your login data in tiny
files called cookies, reducing the need for you re-type your password
when you reconnect to those sites. This means that someone with access
to your computer or phone may be able to access your accounts without
having to steal your password or do sophisticated snooping.

As smart phones have become more popular there has been a dramatic rise
in account hijacking with stolen phones. Laptops theft presents a
similar risk. If you do choose to have the browser save your passwords
then you have a few options to protect yourself:

\begin{itemize}
\item
  Use a screen lock. If you have a phone and prefer an unlock pattern
  system get in the habit of wiping the screen so an attacker can not
  guess the pattern from finger smears. On a Laptop, you should set your
  screensaver to require a password as well as a password on start-up.
\item
  Encrypt your hard disk. TrueCrypt is an open and secure disk
  encryption system for Windows 7/Vista/XP, Mac OS X and Linux. OSX and
  most Linux distributions provide the option for disk encryption on
  install.
\item
  Android Developers: do not enable USB debugging on your phone by
  default. This allows an attacker using the Android \emph{adb shell} on
  a computer to access your phone's hard disk without unlocking the
  phone.
\end{itemize}
\subsection{Can malicious web sites take over my accounts?}

Those special cookies that contain your login data are a primary point
of vulnerability. One particularly popular technique for stealing login
data is called click-jacking, where the user is tricked into clicking on
a seemingly innocuous link, executing a script that takes advantage of
the fact you are logged in. The login data can then be stolen, giving
the remote attacker access to your account. While this is a very
complicated technique, it has proven effective on several occasions.
Both Twitter and Facebook have seen cases of login sessions being stolen
using these techniques.

It's important to develop a habit for thinking before you click on links
to sites while logged into your accounts. One technique is to use
another browser entirely that is not logged into your accounts as a tool
for testing the safety of a link. Always confirm the address (URL) in
the link to make sure it is spelled correctly. It may be a site with a
name very similar to one you already trust. Note that links using URL
shorteners (like http://is.gd and http://bit.ly) present a risk as you
cannot see the actual link you are requesting data from.

If using Firefox on your device, use the add-on
\href{http://noscript.net}{NoScript} as it mitigates many of the
\emph{Cross Site Scripting} techniques that allow for your cookie to be
hijacked but it will disable many fancy features on some web sites.
