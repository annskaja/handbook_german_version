\section{Fears}

\emph{Who can read the email messages that I have already sent or
received?}

\emph{Who can read the emails I send when they travel across the
Internet?}

\emph{Can the people I send emails to share them with anybody?}

Emails that are sent ``in the clear'' without any encryption (which
means the vast majority of email sent and received today) can be read,
logged, and indexed by any server or router along the path the message
travels from sender to receiver. Assuming you use an encrypted
connection (see glossary for TLS/SSL) between your devices and your
email service provider (which everybody should), this means in practice
that the following people can still read any given message:

\begin{enumerate}[1.]
\item
  You
\item
  Your email service provider
\item
  The operators and owners of any intermediate network connections
  (often ambiguous multinational conglomerates or even sovereign states)
\item
  The recipient's email service provider
\item
  The intended recipient
\end{enumerate}
Many webmail providers (like Gmail) automatically inspect all of the
messages sent and received by their users for the purpose of showing
targeted advertisements. While this may be a reasonable compromise for
some users most of the time (free email!), it is disturbing for many
that even their most private communications are inspected and indexed as
part of a hidden and potentially very insightful profile maintained by a
powerful corporate giant with a profit motive.

Additionally, somebody who can legally pressure the groups above could
request or demand:

\begin{enumerate}[1.]
\item
  logged meta-data about email (lists of messages sent or received by
  any user, subject lines, recipients), in some jurisdictions even
  without a warrant.
\item
  messages sent and received by a specific user or group, with a warrant
  or court order in some jurisdictions.
\item
  a dedicated connection to siphon off \emph{all} messages and traffic,
  to be analyzed and indexed off site.
\end{enumerate}
In cases where a user has a business or service relationship with their
email provider, most governments will defend the privacy rights of the
user against unauthorized and unwarranted reading or sharing of
messages, though often it is the government itself seeking information,
and frequently users agree to waive some of these rights as part of
their service agreement. However, when the email provider is the user's
employer or academic institution, privacy rights frequently do not
apply. Depending on jurisdiction, businesses generally have the legal
right to read all of the messages sent and received by their employees,
even personal messages sent after hours or on vacation.

Historically, it was possible to ``get away'' with using clear text
email because the cost and effort to store and index the growing volume
of messages was too high: it was hard enough just to get messages
delivered reliably. This is why many email systems do not contain
mechanisms to preserve the privacy of their contents. Now the cost of
monitoring has dropped much faster than the growth of internet traffic
and large-scale monitoring and indexing of all messages (either on the
sender or receiving side) is reasonable to expect even for the most
innocuous messages and users. {[}CITE:corporate email archiving/spying,
blue coat, Syrian monitoring, USA Utah data center, USA intercept
scandals{]}

For more about legal protections of email messages ``at rest''
(technical term for messages stored on a server after having been
delivered), especially regarding government access to your email
messages, see:

\begin{itemize}
\item
  https://ssd.eff.org/3rdparties/govt/stronger-protection (USA)
\item
  http://en.wikipedia.org/wiki/Data\_Protection\_Directive (EU)
\end{itemize}
Just like there are certain photos, letters, and credentials that you
would not post ``in the clear'' on the Internet because you would not
want that information to get indexed accidentally and show up in search
results, you should never send email messages in the clear that you
would not want an employer or disgruntled airport security officer to
have easy access to.

\subsection{Random abuse and theft by malicious hackers}

\emph{What if somebody gets complete control of my email account?}

\emph{I logged in from an insecure location\ldots{} how do I know now if
my account has been hacked?}

\emph{I've done nothing wrong\ldots{} what do I have to hide?}

\emph{Why would anybody care about me?}

Unfortunately, there are many practical, social, and economic incentives
for malicious hackers to break into the accounts of random Internet
individuals. The most obvious incentive is identity and financial theft,
when the attacker may be trying to get access to credit card numbers,
shopping site credentials, or banking information to steal money. A
hacker has no way to know ahead of time which users might be better
targets than others, so they just try to break into all accounts, even
if the user doesn't have anything to take or is careful not to expose
his information.

Less obvious are attacks to gain access to valid and trusted user
accounts to collect contact email addresses from and then distribute
mass spam, or to gain access to particular services tied to an email
account, or to use as a ``stepping stone'' in sophisticated social
engineering attacks. For example, once in control of your account a
hacker could rapidly send emails to your associates or co-workers
requesting emergency access to more secured computer systems.

A final unexpected problem affecting even low-profile email users, is
the mass hijacking of accounts on large service providers, when hackers
gain access to the hosting infrastructure itself and extract passwords
and private information in large chunks, then sell or publish lists of
login information in online markets.

\subsection{Targeted abuse, harassment, and spying}

\emph{Something I wrote infuriated a person in power\ldots{} how do I
protect myself?}

If you find yourself the individual target of attention from powerful
organizations, governments, or determined individuals, then the same
techniques and principles will apply to keeping your email safe and
private, but additional care must be taken to protect against hackers
who might use sophisticated techniques to undermine your devices and
accounts. If a hacker gains control of any of your computing devices or
gets access to any of your email accounts, they will likely gain
immediate access both to all of your correspondence, and to any external
services linked to your email account.

Efforts to protect against such attacks can quickly escalate into a
battle of wills and resources, but a few basic guidelines can go a long
way. Use specific devices for specific communication tasks, and use them
only for those tasks. Log out and shutdown your devices immediately when
you are done using them. It is best to use open software encryption
tools, web browsers, and operating systems as they can be publicly
reviewed for security problems and keep up to date with security fixes.

\emph{Be wary of opening PDF files using Adobe Reader or other
proprietary PDF readers.} Closed source PDF readers have been known to
be used to execute malign code embedded in the PDF body. If you receive
a .pdf as an attachment you should first consider if you know the
supposed sender and if you are expecting a document from them. Secondly,
you can use PDF readers which have been tested for known vulnerabilities
and do not execute code via java script.

Linux: Evince, Sumatra PDF

OS X: Preview

Windows: Evince

Use short-term anonymous throw away accounts with randomly generated
passwords whenever possible.

\subsection{When Encryption Goes Wrong}

\emph{What happens if I lose my ``keys''? Do I lose my email?}

Rigorous GPG encryption of email is not without its own problems.

If you store your email encrypted and lose all copies of your private
key, you will be absolutely unable to read the old stored emails, and if
you do not have a copy of your revocation certificate for the private
key it could be difficult to prove that any new key you generate is
truly the valid one, at least until the original private key expires.

If you sign a message with your private key, you will have great
difficulty convincing anybody that you did not sign if the recipient of
the message ever reveals the message and signature publicly. The term
for this is \emph{non-repudiation}: any message you send signed is
excellent evidence in court. Relatedly, if your private key is ever
compromised, it could be used to read all encrypted messages ever sent
to you using your public key: the messages may be safe when they are in
transit and just when they are received, but any copies are a liability
and a gamble that the private key will never be revealed. In particular,
even if you destroy every message just after reading it, anybody who
snooped the message on the wire would keep a copy and attempt to decrypt
it later if they obtained the private key.

The solution is to use a messaging protocol that provides \emph{perfect
forward secrecy} by generating a new unique session key for every
conversation of exchange of messages in a random way such that the
session keys could not be re-generated after the fact even if the
private keys were known. The OTR chat protocol provides perfect forward
secrecy
(\href{http://en.wikipedia.org/wiki/Perfect\_forward\_secrecy}{http://en.wikipedia.org/wiki/Perfect\_forward\_secrecy})
for real time instant messaging, and the SSH protocol provides it for
remote shell connections, but there is no equivalent system for email at
this time.

It can be difficult to balance the convenience of mobile access to your
private keys with the fact that mobile devices are much more likely to
be lost, stolen, or inspected and exploited than stationary machines. An
emergency or unexpected time of need might be exactly the moment when
you would most want to send a confidential message or a signed message
to verify your identity, but these are also the moments when you might
be without access to your private keys if your mobile device was seized
or not loaded with all your keys.
