\section{Getting, setting-up and testing a VPN account}

In all the VPN systems, there is one computer set up as a server (in an
unrestricted location), to which one or more clients connect. The set up
of the server is out of the scope of this manual and the set up of this
system is in general covered by your VPN provider. This server is one of
the two ends of the encrypted tunnel. It is important that the company
running this server can be trusted and is located in an area you trust.
So to run a VPN, an account is needed at such a trusted server.

Please keep in mind that an account can often only be used on one device
at a time. If you want to use a VPN with both your mobile and laptop
concurrently, it is very well possible you need two accounts.

\subsection{An account from a commercial VPN provider}

There are multiple VPN providers out there. Some will give you free
trial time, others will begin charging right away at an approximate rate
of €5 per month. Look for a VPN provider that offers OpenVPN accounts -
it is an Open Source, trusted solution available for Linux, OS X, and
Windows, as well as Android and iOS.

When choosing a VPN provider you need to consider the following points:

\begin{itemize}
\item
  Information that is required from you to register an account - the
  less that is needed the better. A truly privacy concerned VPN provider
  would only ask you for email address (make a temporary one!), username
  and password. More isn't required unless the provider creates a user
  database which you probably don't want to be a part of.
\item
  Payment method to be used to pay for your subscription. Cash-transfer
  is probably the most privacy-prone method, since it does not link your
  bank account and your VPN network ID. Paypal can also be an acceptable
  option assuming that you can register and use a temporary account for
  every payment. Payment via a bank transfer or by a credit card can
  severely undermine your anonymity on and beyond the VPN.
\item
  Avoid VPN providers that require you to install their own proprietary
  client software. There is a perfect open source solution for any
  platform, and having to run a ``special'' client is a clear sign of a
  phony service.
\item
  Avoid using PPTP based VPNs, as several security vulnerabilities exist
  in that protocol. In fact, if two providers are otherwise equal,
  choose the one \emph{not} offering PPTP if feasible.
\item
  Look for a VPN provider that's using OpenVPN - an open source,
  multi-platform VPN solution.
\item
  Exit gateways in countries of your interest. Having a choice of
  several countries allows you to change your geo-political context and
  appears to come from a different part of the world. You need to be
  aware of legislation details and privacy laws in that particular
  country.
\item
  Anonymity policy regarding your traffic - a safe VPN provider will
  have a non-disclosure policy. Personal information, such as username
  and times of connection, should not be logged either.
\item
  Allowed protocols to use within VPN and protocols that are routed to
  the Internet. You probably want most of the protocols to be available
\item
  Price vs.~quality of the service and its reliability.
\item
  Any known issues in regard to anonymity of the users the VPN provider
  might have had in the past. Look online, read forums and ask around.
  Don't be tempted by unknown, new, cheap or dodgy offers.
\end{itemize}
There are several VPN review oriented places online that can help you
make the right choice:

\begin{itemize}
\item
  http://www.bestvpnservice.com/vpn-providers.php
\item
  http://vpncreative.com/complete-list-of-vpn-providers
\item
  http://en.cship.org/wiki/VPN
\end{itemize}
Setting up your VPN client

\begin{quote}
``OpenVPN {[}..{]} is a full featured SSL VPN software solution that
integrates OpenVPN server capabilities, enterprise management
capabilities, simplified OpenVPN Connect UI, and OpenVPN Client software
packages that accommodate GNu/Linux, OSX, Windows and environments.
OpenVPN Access Server supports a wide range of configurations, including
secure and granular remote access to internal network and/or private
cloud network resources and applications with fine-grained access
control.''
(\href{http://openvpn.net/index.php/access-server/overview.html}{http://openvpn.net/index.php/access-server/overview.html})

\end{quote}
There is a number of different standards for setting up VPNs, including
PPTP, LL2P/IPSec and \textbf{OpenVPN}. They vary in complexity, the
level of security they provide, and which operating systems they are
available for. Do not use PPTP as it has several security
vulnerabilities. In this text we will concentrate on OpenVPN. It works
on most versions of GNU/Linux, OSX, Windows. OpenVPN is TLS/SSL-based -
it uses the same type of \textbf{encryption} that is used in HTTPS
(Secure HTTP) and a myriad of other encrypted protocols. OpenVPN
encryption is based on \textbf{RSA} key exchange algorithm. For this to
work and in order to communicate, both the server and the client need to
have public and private RSA keys.

Once you obtain access to your VPN account the server generates those
keys and you simply need to download those from the website of your VPN
provider or have them sent to your email address. Together with your
keys you will receive a \emph{root certificate (*.ca)} and a \emph{main
configuration file (*.conf or *.ovpn)}. In most cases only the following
files will be needed to configure and run an OpenVPN client:

\begin{itemize}
\item
  \textbf{client.conf} (or client.ovpn) - configuration file that
  includes all necessary parameters and settings. NOTE: in some cases
  certificates and keys can come embedded inside the main configuration
  file. In such a case the below mentioned files are not necessary.
\item
  \textbf{ca.crt} (unless in configuration file) - root authority
  certificate of your VPN server, used to sign and check other keys
  issued by the provider.
\item
  \textbf{client.crt} (unless in configuration file) - your client
  certificate, allows you to communicate with VPN server.
\end{itemize}
Based on a particular configuration, your VPN provider might require a
username/password pair to authenticate your connection. Often, for
convenience, the username and password can be saved into a separate file
or added to the main configuration file. In other cases, key-based
authentication is used, and the key is stored in a separate file:

\begin{itemize}
\item
  \textbf{client.key} (unless in configuration file) - client
  authentication key, used to authenticate to the VPN server and
  establish an encrypted data channel.
\end{itemize}
In most cases, unless otherwise necessary, you don't need to change
anything in the configuration file and (surely!) \textbf{do not edit key
or certificate files!} All VPN providers have thorough instructions
regarding the setup. Read and follow those guidelines to make sure your
VPN client is configured correctly.

NOTE: Usually it's only allowed to use one key per one connection, so
you probably shouldn't be using the same keys on different devices at
the same time. Get a new set of keys for each device you plan to use
with a VPN, or attempt to set up a local VPN gateway (advanced, not
covered here).

Download your OpenVPN configuration and key files copy them to a safe
place and proceed to the following chapter.

\subsection{Setting up OpenVPN client}

In the following chapters some examples are given for setting up OpenVPN
client software. On any flavor of GNU/Linux use your favorite package
manager and install \textbf{openvpn} or \textbf{openvpn-client} package.

If you want to use OpenVPN on Windows or OSX, have look at:

\begin{itemize}
\item
  \href{http://openvpn.se}{http://openvpn.se} (Windows interface)
\item
  \href{http://code.google.com/p/tunnelblick}{http://code.google.com/p/tunnelblick}
  (OSX interface)
\end{itemize}
