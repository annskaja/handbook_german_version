\section{Basic Tips}

Just as with other forms of communication on the web, some basic
precautions always ought to be taken to ensure you have the best chance
at protecting your privacy.

\subsection{In brief:}

\begin{itemize}
\item
  Passwords shouldn't relate to personal details and should contain a
  mix of more than 8 letters and other characters.
\item
  Always be sure your connection is secure when reading email on a
  wireless network, especially in Internet cafes.
\item
  Temporary files (the `cache') on the computer that you use to check
  your email can present some risks. Clear them often.
\item
  Create and maintain separate email accounts for different tasks and
  interests.
\item
  Encrypt any message you wouldn't feel comfortable sending on a post
  card.
\item
  Be aware of the risks of having your email hosted by a company or
  organization.
\end{itemize}
\subsection{Passwords}

Passwords are a primary point of vulnerability in email communication.
Even a secure password can be read in transit unless the connection is
secure (see TLS/SSL in the glossary). In addition, just because a
password is long doesn't mean it cannot be guessed by using knowledge of
you and your life to determine likely words and numbers.

The general rule for creating passwords is that it should be long (8
characters or more) and have a mix of letters and other characters
(numbers and symbols, which means you could just choose a short
sentence). Combining your birthday with that of a family name is however
a great example of how not to do it. This kind of information is easy to
find using public resources. A popular trick is to base it on a
favourite phrase and then, just to throw people off, sprinkle it with a
few numbers. Best of all is to use a password generator, either on your
local system or online.

Often such passwords are difficult to remember and a second point of
vulnerability is opened up -- physical discovery. Since there is no
better means of storing a password than in your own brain, services like
OnlinePasswordGenerator
(\href{http://www.onlinepasswordgenerator.com/}{http://www.onlinepasswordgenerator.com/})
offer a great compromise by randomly generating passwords that vaguely
resemble words and present you with a list to choose from.

If you do choose to store your password outside your head, you have the
choice to either write it down or use keychain software. This can be a
risky decision, especially if the email account and password are on the
same device like your phone or computer.

Keychain software, like Keepass, consolidates various passwords and
passphrases in one place and makes them accessible through a master
password or passphrase. This puts a lot of pressure on the master
password. If you do decide to use a keychain software, remember to
choose a secure password.

Finally, you should use a different password for different accounts. In
that way, if one of them gets hijacked, your other accounts remain safe.
Never use the same password for your work and private email accounts.
See section \textbf{Passwords} to learn more about how to secure
yourself.

\subsection{Reading Email in Public Places}

One of the great conveniences of wireless networking and `cloud
computing' is the ability to work anywhere. You may often want to check
your email in an Internet cafe or public location. Spies, criminals and
mischievous types are known to visit these locations in order to take
advantage of the rich opportunities offered for ID theft, email snooping
and hijacking bank accounts.

Here we find ourselves within an often underestimated risk of someone
listening in on your communications using \emph{network packet
sniffing}. It matters little if the network itself is open or password
secured. If someone joins the same encrypted network, s/he can easily
capture and read all \emph{unsecured} (see chapter \textbf{Secure
Connection}) traffic of all of other users within the same network. A
wireless key can be acquired for the cost of a cup of coffee and gives
those that know how to capture and read network packets the chance to
read your password while you check your email.

Here a simple general rule always applies: if the cafe offers a network
cable connection, use it! Finally, just as at a bank machine, make sure
no one watches over your shoulder when you type in the password.

\subsection{Cache Cunning}

Here again convenience quickly paves the road to bad places. Due to the
general annoyance of having to type in your password over and over
again, you ask the browser or local mail client to store it for you.
This is not bad in itself, but when a laptop or phone gets stolen, it
enables the thief to access the owner's email account(s). The best
practice is to clear this cache every time you close your browser. All
popular browsers have an option to clear this cache on exit.

One basic precaution can justify you holding onto your convenient cache:
disk encryption. If your laptop is stolen and the thief reboots the
machine, they'll be met with an encrypted disk. It is also wise to have
a screen lock installed on your computer or phone. If the machine is
taken from you while still running your existing browsing session, it
cannot be accessed. Securing your communication

Whenever you write and send email in a browser or use an email program
(Outlook Express, Mozilla Thunderbird, Mail.app or Mutt), you should
always ensure to use encryption for the entire session. This is easily
done due to the popular use of \emph{TLS/SSL (Secure Socket Layer)}
connections by email servers (See glossary \textbf{TLS/SSL}).

If using a browser to check your email, check to see if the mail server
supports SSL sessions by looking for https:// at the beginning of the
URL. If not, be sure to turn it on in your email account settings, such
as Gmail or Hotmail.This ensures that not just the login part of your
email session is encrypted but also the writing and sending of emails.

At the time of writing, Google's Gmail uses TLS/SSL by default whereas
Hotmail does not. If your email service does not appear to provide
TLS/SSL, then it is advised to stop using it. Even if your emails are
not important, you might find yourself `locked out' of your account one
day with a changed password!

When using an email program to check your email, be sure that you are
using TLS/SSL in the program options. For instance in Mozilla
Thunderbird the option for securing your outgoing email is found in
\verb!Tools -> Account Settings -> Outgoing Server (SMTP)! and for
incoming email in \verb!Tools -> Account Settings -> Server Settings!.
This ensures that the downloading and sending of email is encrypted,
making it very difficult for someone on your network, or on any of the
networks between you and the server, to read or log your email.
Encrypting the email itself

Even if the line itself is encrypted using a system such as SSL, the
email service provider still has full access to the email because they
own and have full access to the storage device where you host your
email. If you want to use a web service and be sure that your provider
cannot read your messages, then you'll need to use something like
\emph{GPG} (Appendix for \textbf{GnuPG}) with which you can encrypt the
email. The header of the email however will still contain the IP
(Internet address) that the email was sent from alongside other
compromising details. Worth mentioning here is that the use of
\emph{GPG} in webmail is not as comfortable as with a locally installed
mail client, such as \emph{Thunderbird} or \emph{Outlook Express}.

\subsection{Account Separation}

Due to the convenience of services like Gmail, it is increasingly
typical for people to have only one email account. This considerably
centralises the potential damage done by a compromised account. More so,
there is nothing to stop a disgruntled Google employee from deleting or
stealing your email, let alone Google itself getting hacked. Hacks
happen.

A practical strategy is to keep your personal email, well, personal. If
you have a work email then create a new account if your employers
haven't already done it for you. The same should go for any clubs or
organisations you belong to, each with a unique password. Not only does
this improve security, by reducing the risk of whole identity theft, but
greatly reduces the likelihood of spam dominating your daily email.

\subsection{A note about hosted email}

Those that provide you with the service to host, send, download and read
email are not encumbered by the use of TLS/SSL. As hosts, they can read
and log your email in plain text. They can comply with requests by local
law enforcement agencies who wish to access email. They may also study
your email for patterns, keywords or signs of sentiment for or against
brands, ideologies or political groups. It is important to read the EULA
(End-user license agreement) of your email service provider and do some
background research on their affiliations and interests before choosing
what kind of email content they have access to. These concerns also
apply to the hosts of your messages' recipients.
