\section{How To CryptoParty}

\begin{itemize}
\item
  Throw a party. All you need is a time, a date and a location. Add it
  to the wiki: \href{https://cryptoparty.org}{https://cryptoparty.org}.
\item
  Make sure you have Internet connectivity and enough power sources for
  all devices. If you do not have a place to hold a CryptoParty, find a
  pub or park where you can meet and squeeze the public bandwidth. That
  will really hone your skills!
\item
  Bring USB sticks and printed handouts for those who need them, and set
  up old computers for people to fiddle with and try out new skills.
\item
  Talk about Linux to everyone you meet at your CryptoParty. If you are
  new to CryptoParties - ask someone ``what is Linux?'' ASAP.
\item
  Make entry free for all if possible - CryptoParties are
  not-for-profit, not commercially aligned and especially important for
  those without other resources.
\item
  Teach basic cryptographic tools to the masses. Crowd-source the best
  crypto. We suggest PGP, OTR, and Tor as the first tools to install.
\item
  Invite experts and non-experts from all fields. Everyone is an expert
  on something.
\item
  If you want CryptoParty to do something, start doing it. Organise
  organically and chaotically. Have no clear leadership. Urge people to
  take on a sudo leadership role - take a tutorial, fix the wifi, update
  the wiki, or organise the next CryptoParty. If someone claims others
  are doing it wrong - invite them to nominate themselves to do it
  better.
\item
  Ask for feedback. Assimilate critics - ask them for their help in
  creating a better CryptoParty. Do not be scared to troll the trolls
  back or boot them from your space. Share feedback on the wiki.
  Iterate.
\item
  A successful CryptoParty can have as many or as few as two people.
  Size doesn't count, it's what you do with it that matters. The
  criterion for success should be that everyone had fun, learned
  something and wants to come to the next party.
\item
  Think of the CryptoParty movement as a huge Twitter hive ready to
  swarm at any moment. Tweet a lot, and make your tweets are meaningful.
  ReTweet other CryptoPartiers frequently.
\item
  Make sure the way crypto is taught at your party could be understood
  by a 10 year old. Then have the 10 year old teach it to an 80 year
  old. Breach the digital divide with random acts of awesomeness such as
  unfettered use of images of kittens in all CryptoParty literature. Red
  underpants on heads is only mandatory if you wish to bid in our
  spectrum auction.
\item
  Consider hosting private, off-the-radar CryptoParties for activists,
  journalists and in individuals working in dangerous locations.
\item
  Don't scare non-technical people. Don't teach command lines before
  people know where the on-off buttons are located on their laptops.
  Everyone learns at their own pace - make sure there is support for
  those in need of help.
\item
  Doing excellent stuff at CryptoParty does not require permission or an
  official consensus decision. If you're uncertain about the excellence
  of something you want to do, you should ask someone else what they
  think.
\item
  Consider the need for a bouncer, particularly if your CryptoParty
  expects over 50 people. Dress the bouncer up as a Sumo wrestler. Do
  not be afraid bounce people who breach CryptoParty's anti-harassment
  policy.
\item
  CryptoParty is dedicated to providing a harassment-free sharing
  experience for everyone, regardless of gender, sexual orientation,
  disability, physical appearance, body size, heritage, or religion.
  Behaving like an arsehole may mean you are permanently uninvited to
  CryptoParties events. Harassment includes:

  \begin{itemize}
  \item
    hurtful or offensive comments
  \item
    deliberate intimidation
  \item
    direct or indirect threats
  \item
    stalking
  \item
    following
  \item
    inappropriate physical contact
  \item
    unwelcome sexual attention.
  \end{itemize}
\item
  Encourage a culture of sharing. Encourage advanced users to help
  not-so advanced ones. Delegate.
\item
  Use online meeting platforms like mumble (e.g. \#cryptoparty room on
  http://occupytalk.org/) when physical meetups are not possible or
  impractical.
\item
  Copy from other cryptoparties. Remix, Reuse and Share. Create a basket
  of old devices people are willing to donate to more needy
  CryptoPartiers.
\item
  Get the word out! Print posters and/or flyers and distribute them in
  your neighbourhood, post online versions to social networks and mail
  them to friends, for them to distribute the info even further.
\item
  Don't sell out to sponsors for pizza and beer money. Ask people to try
  and bring food and drink to share. Host CryptoPicnics as often as
  possible. Make friends with librarians. They wield power over keys to
  local, public meeting rooms that may be free of charge to utilize.
\item
  Invite all the people. Bring people together who have a wide range of
  skills and interests - musicians, political pundits, activists,
  hackers, programmers, journalists, artists and philosophers. Spread
  the love.
\item
  Invite the graphic designers and illustrators you know to contribute
  new ways to help people understand crypto.
\item
  Invite everyone to share their knowledge and their skills. Individuals
  with little or no coding, programming, hacking or crypto skills can
  change cultures by promoting the idea that privacy is a fundamental
  right.
\item
  Share music, beers, \& chips. Bond together over eclectic music,
  cheeseballs, installing GPG, TrueCrypt, OTR and Tor, as well as
  watching movies together. We recommend Hackers, The Matrix,
  Bladerunner, Tron, Wargames, Sneakers, and The Net.
\item
  Do not work too hard. Take breaks. Eat popcorn together. Create slang,
  phrases, memes.
\item
  When people at CryptoParties ask for advice on ``hacking the Gibson''
  refer them to episodes of `My Little Pony'.
\item
  Create fliers and advertise using slogans like: ``CryptoParties: If
  there is hope, it lies in the proles'' and ``CryptoParty like it's
  1984.'' CryptoParty all the things to avoid oppression and depression.
\item
  Seed CryptoParties in your local communities - at nursing homes, scout
  groups, music festivals, universities, schools. Take CryptoParty to
  isolated and remote communities. Make friends in far away places and
  travel whenever possible. Ask people in rural farming communities if
  they'd like to CryptoParty.
\item
  Share shimmering opportunities of crowd-sourced privacy: swap cheap,
  pre-paid SIMs, handsets and travel cards.
\item
  Create logos in bright pink and purple, with hearts all over them.
  Promote CryptoParties to rebellious 13 year old girls. Declare success
  if rebellious 13 year old girls demand to attend your parties.
\item
  Become friends with journalists. Invite them to your parties. Teach
  them crypto. Do not scare them by discussing Assassination Markets.
\item
  Strew CryptoParty sigils across your city in 3am post-party raids.
  Make lots of stickers, paste them everywhere.
\item
  Experiment, constantly. Do not be afraid to make mistakes. Encourage
  people to tinker. Assume all mistakes are meant to made. Most people
  under intel agency scrutiny have electronic devices already
  compromised before they walk in the door. Teach people to install
  tools from scratch, so they can do it on a new machine, away from
  prying eyes.
\item
  Assume intel agencies send representative to CryptoParties.
  Acknowledge their presence at the start of your meeting, ask them to
  share their crypto skills. Joke about paranoia as often as possible
  without instilling panic. Wear tinfoil hats.
\item
  Be excellent to each other and CryptoParty on.
\end{itemize}
