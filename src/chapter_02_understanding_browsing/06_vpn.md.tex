\section{VPN}

The way your data makes it to the desired server and back to your laptop
computer or a mobile device is not as straightforward as it might first
seem. Say, you are connected to a wireless network at home and opening a
wikipedia.org page. The path your request (data) takes will consist of
multiple middle points or \emph{``hops''} - in network-architect
terminology. At each of these hops (which are likely to be more then 5)
your data can be scooped, copied and potentially modified.

\begin{itemize}
\item
  Your wireless network (your data can be sniffed from the air)
\item
  Your ISP (in most countries they are obliged to keep detailed logs of
  user activity)
\item
  Internet Exchange Point (IXP) somewhere on another continent (usually
  more secure then any other \emph{hop})
\item
  ISP of the hosting company that hosts the site (is probably keeping
  logs)
\item
  Internal network to which the server is connected
\item
  And multiple hops between\ldots{}
\end{itemize}
Any person with physical access to the computers or the networks which
are on the way from you to the remote server, intentionally or not, can
collect and reveal the data that's passing from you to the remote server
and back. This is especially true for so called `last mile' situations -
the few last leaps that an internet connection makes to reach a user.
That includes domestic and public wireless or wired networks, telephone
and mobile networks, networks in libraries, homes, schools, hotels. Your
ISP can not be considered a safe, or `data-neutral' instance either - in
many countries state agencies do not even require a warrant to access
your data, and there is always the risk of intrusion by paid attackers
working for a deep-pocketed adversaries.

VPN - a Virtual Private Network - is a solution for this `last-mile'
leakage. VPN is a technology that allows the creation of a virtual
network on top of an existing infrastructure. Such a VPN network
operates using the same protocols and standards as the underlying
physical network. Programs and OS use it transparently, as if it was a
separate network connection, yet its topology or the way how network
nodes (you, the VPN server and, potentially, other members or services
available on VPN) are interconnected in relation to the physical space
is entirely redefined.

Imagine that instead of having to trust your data to every single
middle-man (your local network, ISP, the state) you have a choice to
pass it via a server of a VPN provider whom you trust (after a
recommendation or research) - from which your data will start its
journey to the remote location. VPN allows you to recreate your local
and geo-political context all together - from the moment your data
leaves your computer and gets into the VPN network it is fully secured
with TLS/SSL type encryption. And as such it will appear as pure random
noise to any node who might be spying after you. It is as if your data
was traveling inside a titanium-alloy pipe, unbreakable on all the way
from your laptop to the VPN server. Of course one could argue that
eventually, when your data is outside the safe harbour of VPN it becomes
just as vulnerable as it was - but this is only partially true. Once
your data exits the VPN server it is far away from you - way beyond the
reach of some creeps sniffing on the local wireless network, your venal
ISP or a local government obsessed with anti-terrorism laws. A serious
VPN provider would have their servers installed at a high-security
Internet exchange location, rendering any physical human access, tapping
or logging a difficult task.

\begin{quote}
``Today everything you do on the Internet is monitored and we want to
change that. With our fast VPN service you get totally anonymous on the
Internet. It's also possible to surf censored web sites, that your
school, ISP, work or country are blocking. {[}DarkVPN{]} will not only
help people to surf anonymously, it also helps people in countries like
China to be able to surf censored web pages. Which is your democratic
right. DarknetVPN gives all VPN users an anonymous IP address. All
electronic tracks will end up with us. We do not save any log files in
order to achieve maximum anonymity. With us you always surfing
anonymously, secure and encrypted.''
(\href{http://www.darknetvpn.com/about.php}{http://www.darknetvpn.com/about.php})

\end{quote}
Another interesting and often underrated features of VPN is encoded in
its name - besides being \textbf{V}irtual and \textbf{P}rivate it is
also a \textbf{N}etwork. VPN allows one not only to connect via the VPN
server to the rest of the world but also to communicate to other members
of the same VPN network without ever having to leave the safety of
encrypted space. Through this functionality Virtual Private Network
becomes something like a \emph{DarkNet} (in a broader sense of the
definition) - a network isolated from the Internet and inaccessible to
``others''. Since a connection to VPN server, and thus the private
network it facilitates, require a key or a \emph{certificate}, only
``invited'' users are allowed. There is no chance that Internet stranger
would gain access to what's on a VPN without enrolling as a user or
stealing someones keys. While not referred to as such, any corporate
Intranet type of network is a DarkNet too.

\begin{quote}
``A virtual private network (VPN) is a technology for using the Internet
or another intermediate network to connect computers to isolated remote
computer networks that would otherwise be
inaccessible..''(\href{http://en.wikipedia.org/wiki/Virtual\_private\_network}{http://en.wikipedia.org/wiki/Virtual\_private\_network})

\end{quote}
Many commercial VPN providers stress the anonymity that their service
provides. Quoting Ipredator.org page (a VPN service started by the
people behind The Pirate Bay project): ``You'll exchange the IP address
you get from your ISP for an anonymous IP address. You get a
safe/encrypted connection between your computer and the Internet''.
Indeed, when you access the Internet via a VPN connection it does appear
as if the connection is originating from the IP address of IPredator
servers.

\begin{quote}
``You'll exchange the IP address you get from your ISP for an anonymous
IP address. You get a safe/encrypted connection between your computer
and the Internet.''
(\href{https://www.ipredator.se}{https://www.ipredator.se})

\end{quote}
