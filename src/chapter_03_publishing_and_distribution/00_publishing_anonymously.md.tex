\section{Publishing Anonymously}

Whether you are an activist operating under a totalitarian regime, an
employee determined to expose some wrongdoings in your company or a
vengeful writer composing a bitchy portrait of your ex-wife, you need to
protect your identity. If you are not collaborating with others, the
focus lies on anonymity and not encryption or privacy.

If the message is urgent and the stakes are high, one easy way to just
get it out quickly is going to an internet cafe one usually does not
frequent, create accounts specifically set up for the task, deliver the
data and discard those accounts right after that. If you are in a hurry,
consider MintEmail
(\href{http://www.mintemail.com/}{http://www.mintemail.com/}) or
FilzMail (\href{http://www.filzmail.com/}{http://www.filzmail.com/}),
where your address will expire from 3 to 24 hours respectively. Do not
do anything else while you're there; don't check your gmail account, do
not have a quick one on Facebook and clear all cache, cookies and
history and close the browser before you leave.

If you keep these basic rules, the worst -- though highly improbable --
thing that could happen would be that the offered computer is
compromised and logging keystrokes, revealing passwords or even your
face, in case an attached webcam is remotely operated. Don't do this at
work or in a place where you are a registered member or a regular
visitor, like a club or a library.

If you want to maintain a constant stream of communication and maybe
even establish an audience, this method quickly becomes quite
cumbersome, and you might also run out of unused internet cafes. In this
case you can use a machine you own, but, if you cannot dedicate one
especially to this purpose, boot your computer with a different
operating system (OS). This can be easily done by using a USB stick to
boot a live operating system like TAILS, which comes with TOR enabled by
default and includes state-of-the-art cryptographic tools. In any case,
use Tor to disguise your IP.

Turn off all cookies, history and cache options and never use the same
profile or the same browser for other activities. Not only would that
add data to your topography as a user in the Net, but it also opens a
very wide window for mistakes. If you want extra support, install
\emph{Do Not Track Plus} and \emph{Trackerblock} or \emph{Ghostery} in
your browser add-ons menu.

Use passwords for different accounts and choose proper passwords or even
passphrases (more about that in the basic tips section). Protect your
entire system with a general password, change it often and do not share
it with anyone, \emph{especially} not your lover. Install a keystroke
logger to see if someone sneaks into your email, especially your lover.
Set up your preferences everywhere to log out of every service and
platform after 5 minutes of non-use. Keep your superhero identity to
yourself.

If you can mantain such level of discipline, you should even be capable
of using your own internet connection. But consider this: not using a
dedicated system makes it incredibly difficult to keep all the different
identities separated in a safe way, and the feeling of safety often
leads to carelessness. Keep a healthy level of neurosis.

Today there are many publishing possibilities, from cost-free blogging
sites (Blogspot, Tumblr, WordPress, Identi.ca) to PasteBins (see
glossary) and some specifically catered to anonymous users like
BlogACause. Global Voices Advocacy recommends using WordPress through
the Tor network. Keep a sane level of cynicism; they all act in
commercial interests that you use for `free' and so cannot be trusted at
all, especially in that they may be bound to the demands of a legal
juristiction that is not your own. All providers are, when it comes down
to it, traitors.

If registration with these services requires a working email address,
create one dedicated solely to this purpose. Avoid Gmail, Yahoo, Hotmail
and other big commercial platforms with a history of turning over their
users and go for an specialized service like Hushmail
(\href{https://www.hushmail.com/}{https://www.hushmail.com/}). For more
on anonymous email, please find the chapter Anonymous email in the
previous section.

\subsection{Several Don'ts}

\textbf{Don't register a domain.} There are services that will protect
your identity from a simple who is query, like Anonymous Speech or
Silent Register, but they will know who you are through your payment
data. Unless you have the chance to purchase one in BitCoins, limit
yourself to one of the domains offered by your blogging platform like
yourblogname.blogspot.com and choose a setting outside your native
country. Also, find a name that doesn't give you away easily. If you
have problems with that, use a blog name generator online.

\textbf{Don't open a social network account associated to your blog.} If
you must, keep the level of hygiene that you keep for blogging and never
ever login while using your regular browser. If you have a public social
network life, avoid it all together. You will eventually make a mistake.

\textbf{Don't upload video, photo or audio files} without using an
editor to modify or erase all the meta data (photos contain information
up to the GPS coordinates of the location the photo was taken at) that
standard digital cameras, SmartPhones, recorders and other devices add
by default. The \emph{Metadata Anonymisation Toolkit} might help you
with that.

\textbf{Don't leave a history.} Add X-Robots-Tag to your http headers to
stop the searching spiders from indexing your website. That should
include repositories like the Wayback Machine from archive.org. If you
don't know how to do this, search along the lines of ``Robots Text File
Generator''.

\textbf{Don't leave comments.} If you must, maintain the levels of
hygiene that you use for blogging and always logout when you're done and
for god sakes do not troll around. Hell hath no fury like a blogger
scorned.

\textbf{Don't expect it to last.} If you hit the pot and become a
blogging sensation (like \emph{Belle de Jour}, the British PhD candidate
who became a sensation and sold a book and mused two TV shows about her
double life as a high escort) there will be a legion of journalists, tax
auditors and obsessive fans scrutinizing your every move. You are only
human: they will get to you.

\textbf{Don't linger.} If you realize you have already made any mistakes
but nobody has caught you yet, do close all your accounts, uncover your
tracks and start a totally new identity. The Internet has infinite
memory: one strike, and you're out of the closet.
