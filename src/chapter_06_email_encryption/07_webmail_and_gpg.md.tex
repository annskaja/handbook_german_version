\section{Webmail and PGP}

The only safe way of encrypting email inside of the browser window is to
encypt it outside and then copy \& paste the encrypted text into the
browser window.

For example, write the text in a text editor like gedit, vim or kate and
save it as .txt file (in this example ``message.txt''. Then type

\begin{verbatim}
gpg -ase -r the.recipients.email.address@or.gpg.id -r your.gpg.id message.txt
\end{verbatim}
A new file called ``message.asc'' will be created. It contains the
encrypted message and can thus be either attached to an email or its
content safely copy \& pasted into the browser window.

To decrypt a message from the browser window, simply type \verb!gpg!
into the command line and hit Enter. Then copy \& paste the message to
be decrpyted into the commandline window and after being asked for your
passphrase hit Ctrl+D (this enters a end-of-file character and prompts
gpg to output the cleartext message).

If using the commandline seems too cumbersome to you, you might consider
installing a helper application like gpgApplet, kgpg or whatever
application ships with your operating system.
