\section{Basic Tips}

\subsection{In Brief:}

\begin{itemize}
\item
  When you visit a website you give away information about yourself to
  the site owner, unless precautions are taken.
\item
  Your browsing on the Internet may be tracked by the sites you visit
  and partners of those sites. Use anti-tracking software.
\item
  Visiting a website on the Internet is never a direct connection. Many
  computers, owned by many different people are involved. Use a secure
  connection to ensure your browsing can not be recorded.
\item
  What you search for is of great interest to search providers. Use
  search anonymising software to protect your privacy.
\item
  It is wiser to trust Open Source browsers like Mozilla Firefox as they
  can be more readily security audited.
\end{itemize}
\subsection{Your browser talks about you behind your back}

All browsers communicate information to the web server serving you a web
page. This information includes name and version of the browser,
referral information (a link on another site, for instance) and the
operating system used.

Websites often use this information to customise your browsing
experience, suggesting downloads for your operating system and
formatting the web page to better fit your browser. Naturally however,
this presents an issue as regards the user's own anonymity as this
information becomes part of a larger body of data that can be used to
identify you individually.

Stopping the chatter of your browser is not easily done. You can,
however, falsify some of the information sent to web servers while you
browse by altering data contained in the \emph{User Agent}, the
browser's identity. There is a very useful plugin for Firefox, for
instance, called \emph{User Agent Switcher} that allows you to set the
browser identity to another profile selected from a drop down list of
options.

\subsection{Web sites can track you as you browse}

Small files, called \emph{cookies}, are often written onto your computer
by web sites. Cookies present certain conveniences, like caching login
data, session information and other data that makes your browsing
experience smoother. These small pieces of data however present a
significant risk to your right to anonymity on the web: they can be used
to identify you if you return to a site and also to track you as you
move from site to site. Coupled with the User-Agent, they present a
powerful and covert means of remotely identifying your person.

The ideal solution to this problem is deny all website attempts to write
cookies onto your system but this can greatly reduce the quality of your
experience on the web.

See the section \textbf{Tracking} for guides as to how to stop web
servers tracking you.

\subsection{Searching online can give away information about you}

When we search online using services like Bing or Google our right to
privacy is already at risk, vastly more so than asking a person at an
Information Desk in an airport, for instance.

Combined with the use of cookies and User Agent data this information
can be used to build an evolving portrait of you over time. Advertisers
consider this information very valuable, use it to make assumptions
about your interests and market you products in a targeted fashion.

While some customers may sing the praises of targeted advertising and
others may not care, the risks are often misunderstood. Firstly, the
information collected about you may be requested by a government, even a
government you did not elect (Google, for instance, is an American
company and so must comply with American judicial processes and
political interests). Secondly there is the risk that merely searching
for information can be misconstrued as intent or political endorsement.
For instance an artist studying the aesthetics of different forms of
Religious Extremism might find him or herself in danger of being
associated with support for the organisations studied. Finally there is
the risk that this hidden profile of you may be sold on to insurance
agents, provided to potential employers or other customers of the
company whose search service you are using.

Even once you've ensured your cookies are cleared, your \emph{User
Agent} has been changed (see above and chapter \textbf{Tracking}) you
are still giving away one crucial bit of information: the Internet
Address you are connecting from (see chapter \textbf{What Happens When
You Browse}). To avoid this you can use an anonymising service like Tor
(see chapter \textbf{Anonymity}). If you are a Firefox user
(recommended) be sure to install the excellent \emph{Google Sharing}
add-on, an anonymiser for Google search. Even if you don't consciously
use Google, a vast number of web sites use a customised Google Search
bar as a means of exploring their content.

With the above said, there are no reasons to trust Google, Yahoo or
Bing. We recommend switching to a search service that takes your right
to privacy seriously: DuckDuckGo
(\href{http://duckduckgo.com/}{http://duckduckgo.com/}).

\subsection{More eyes than you can see}

The Internet is a big place and is not one network but a greater network
of many smaller interconnected networks. So it follows that when you
request a page from a server on the Internet your request must traverse
many machines before it reaches the server hosting the page. This
journey is known as a \emph{route} and typically includes at least 10
machines along the path. As packets move from machine to machine they
are necessarily copied into memory, rewritten and passed on.

Each of the machines along a network route belongs to someone, normally
a company or organisation and may be in entirely different countries.
While there are efforts to standardise communication laws across
countries, the situation is currently one of significant jurisdictional
variation. So, while there may not be a law requiring the logging of
your web browsing in your country, such laws may be in place elsewhere
along your packet's route.

The only means of protecting the traffic along your route from being
recorded or tampered with is using \emph{end to end encryption} like
that provided by TLS/Secure Socket Layer (See chapter
\textbf{Encryption}) or a Virtual Private Network (See chapter
\textbf{VPN}).

\subsection{Your right to be unknown}

Beyond the desire to minimise privacy leakage to specific service
providers, you should consider obscuring the Internet Address you are
connecting from more generally (see chapter \textbf{What Happens When
You Browse}). The desire to achieve such anonymity spurred the creation
of the \emph{Tor Project}.

\emph{Tor} uses an ever evolving network of nodes to route your
connection to a site in a way that cannot be traced back to you. It is a
very robust means of ensuring your Internet address cannot be logged by
a remote server. See the chapter \textbf{Anonymity} for more information
about how this works and how to get started with Tor.
