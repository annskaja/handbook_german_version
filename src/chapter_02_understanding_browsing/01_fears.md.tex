\section{Fears}

\subsection{Social Networking - what are the dangers?}

The phenomenon of Internet based Social Networking has changed not just
how people use the Internet but its very shape. Large data centers
around the world, particularly in the US, have been built to cater to
the sudden and vast desire for people to upload content about
themselves, their interests and their lives in order to participate in
Social Networking.

Social Networking as we know it with FaceBook, Twitter (and earlier
MySpace) are certainly far from `free'. Rather, these are businesses
that seek to develop upon, and then exploit, a very basic anxiety: the
fear of social irrelevance. As social animals we can't bear the idea of
missing out and so many find themselves placing their most intimate
expressions onto a businessman's hard-disk, buried deep in a data center
in another country - one they will never be allowed to visit.

Despite this many would argue that the social warmth and personal
validation acquired through engagement with Social Networks well
out-weighs the potential loss of privacy. Such a statement however is
only valid when the \emph{full} extent of the risks are known.

The risks of Social Networking on a person's basic right to privacy are
defined by:

\begin{itemize}
\item
  The scope and intimacy of the user's individual contributions.
\item
  A user posting frequently and including many personal details
  constructs a body of information of greater use for targeted
  marketing.
\item
  The preparedness of the user to take social risks.
\item
  A user making social connections uncritically is at greater risk from
  predators and social engineering attacks.
\item
  The economic interests and partners of the organisation providing the
  service.
\item
  Commissioned studies from clients, data mining, sentiment analysis.
\item
  Political/legal demands exerted by the State against the organisation
  in the jurisdiction(s) in which it is resident.
\item
  Court orders for data on a particular user (whether civilian or
  foreigner).
\item
  Surveillance agendas by law enforcement or partners of the
  organisation.
\item
  Sentiment analysis: projections of political intent.
\end{itemize}
With these things in mind it is possible to chart a sliding scale
between projects like Diaspora and Facebook: the former promises some
level of organisational transparency, a commitment to privacy and a
general openness, whereas Facebook proves to be an opaque company
economically able to gamble with the privacy of their users and manage
civil lawsuits in the interests of looking after their clients. As such
there is more likelihood of your interactions with a large Social
Network service affecting how an Insurance company or potential employer
considers you than a smaller, more transparent company.

\subsection{Who can steal my identity?}

This question depends on the context you are working within as you
browse. A weak and universal password presents a danger of multiple
services from Social Networking, Banking, WebMail etc being account
hijacked. A strong and universal password on a wireless network shared
with others (whether open or encrypted) is just as vulnerable. The
general rule is to ensure you have a strong password (see section on
\textbf{Passwords}).

\subsubsection{Wireless networks}

Here we find ourselves amidst an often underestimated risk of someone
listening in on your communications using \emph{network packet
sniffing}. It matters little if the network itself is open or password
secured. If someone uses the same encrypted network, he can easily
capture and read all unsecured traffic of other users within the same
network. A wireless key can be acquired for the cost of a cup of coffee
and gives those that know how to capture and read network packets the
chance to read your password while you check your email.

A simple rule always applies: if the cafe offers a network cable
connection, use it! Finally, just as at a bank machine, make sure no one
watches over your shoulder when you type in the password.

\subsubsection{The browser cache}

Due to the general annoyance of having to type in your password
repeatedly, you allow the browser or local mail client to store it for
you. This is not bad in itself, but when a laptop or phone gets stolen,
this enables the thief to access the owner's email account(s). The best
practice is to clear this cache every time you close your browser. All
popular browsers have an option to clear this cache on exit.

One precaution can justify you holding onto your convenient cache: disk
encryption. If your laptop is stolen and the thief reboots the machine,
they'll be met with an encrypted disk. It is also wise to have a screen
lock installed on your computer or phone. If the machine is taken from
you while still running your existing user session, it cannot be
accessed.

\subsubsection{Securing your line}

Whenever you log into any service you should always ensure to use
encryption for the entire session. This is easily done due to the
popular use of \emph{TLS/SSL (Secure Socket Layer)}.

Check to see the service you're using (whether Email, Social Networking
or online-banking) supports TLS/SSL sessions by looking for
\verb!https://! at the beginning of the URL. If not, be sure to turn it
on in any settings provided by the service. To better understand how
browsing the World Wide Web works, see the chapter \textbf{What Happens
When I Browse?}

\subsection{Can I get in trouble for Googling weird stuff?}

Google and other search companies may comply with court orders and
warrants targeting certain individuals. A web site using a customised
Google Search field to find content on their site may be forced to log
and supply all search queries to organisations within their local
jurisdiction. Academics, artists and researchers are particularly at
risk of being misunderstood, assumed to have motivations just by virtue
of their apparent interests.

\subsection{Who is keeping a record of my browsing and am I allowed to
hide from them?}

It is absolutely within your basic human rights, and commonly
constitutionally protected, to visit web sites anonymously. Just as
you're allowed to visit a public library, skim through books and put
them back on the shelf without someone noting the pages and titles of
your interest, you are free to browse anonymously on the Internet.

\subsection{How to not reveal my Identity?}

See the chapter on \textbf{Anonymity}.

\subsection{How to avoid being tracked?}

See the chapter on \textbf{Tracking}.
