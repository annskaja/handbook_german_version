\section{Keeping passwords safe}

Passwords are like keys in the physical world. If you lose a password
you will not be able to get in, and if others copy or steal it they can
use it to enter. A good password should not be easy for others to guess
and not easy to crack with computers, while still being easy for you to
remember.

\subsection{Password length and complexity}

To protect your passwords from being guessed, length and complexity are
important. Passwords like the name of your pet or a birth date are very
unsafe, as is using single word that can be found in a dictionary. Do
not use a password containing only numbers. Most importantly a secure
password is long. Using combinations of lower case letters, capitals,
numbers and special characters can improve the security, but length is
still the most important factor.

For use with important accounts like the pass phrase which protects your
PGP/GPG or TrueCrypt encrypted data, or the password for your main email
account, use 20 characters or more, the longer the better. See
\href{https://xkcd.com/936/}{this XKCD cartoon}
\verb!"correct horse battery staple"! vis-à-vis \verb!"Tr0ub4dor&3"! for
an explanation.

\subsection{Easy to remember and secure passwords}

One way to create strong and easy to remember passwords is to use
sentences.

A few examples:

\begin{itemize}
\item
  \verb!IloveDouglasAdamsbecausehe'sreallyawesome.!
\item
  \verb!Peoplelovemachinesin2029A.D.!
\item
  \verb"BarneyfromHowIMetYourMotherisAWESOME!"
\end{itemize}
Sentences are easy to remember, even if they are 50 characters long and
contain uppercase characters, lowercase characters, symbols and numbers.

\subsection{Minimizing damage}

It is important to minimize the damage if one of your passwords is ever
compromised. Use different passwords for different websites or accounts,
that way if one is compromised, the others are not. Change your
passwords from time to time, especially for accounts you consider to be
sensitive. By doing this you can block access to an attacker who may
have learned your old password.

\subsection{Using a password manager}

Remembering a lot of different passwords can be difficult. One solution
is to use a dedicated application to manage most of your passwords. The
next section in this chapter will discuss \emph{Keepass}, a free and
open source password manager with no known vulnerabilities, so long as
you chose a sufficiently long and complex ``master password'' to secure
it with.

For website passwords only, another option is the built-in password
manager of the Firefox browser. Make sure to set a master password,
otherwise this is very insecure!

\subsection{Physical protection}

When using a public computer such as at a library, an internet cafe, or
any computer you do not own, there are several dangers. Using ``over the
shoulder'' surveillance, someone, possibly with a camera, can watch your
actions and may see the account you log in to and the password you type.
A less obvious threat is software programs or hardware devices called
``keystroke loggers'' that record what you type. They can be hidden
inside a computer or a keyboard and are not easily spotted. Do not use
public computers to log in to your private accounts, such as email. If
you do, change your passwords as soon as you get back to a computer you
own and trust.

\subsection{Other caveats}

Some applications such as chat or mail programs may ask you to save or
``remember'' your username and password, so that you don't have to type
them every time the program is opened. Doing so may mean that your
password can be retrieved by other programs running on the machine, or
directly from your hard disk by someone with physical access to it.

If your login information is sent over an insecure connection or
channel, it might fall into the wrong hands. See the chapters on secure
browsing for more information.
