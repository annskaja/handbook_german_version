\section{File Sharing}

The term \emph{File Sharing} refers to the practice of sharing files on
a network, often with widest possible distribution in mind.
Unfortunately in recent years the term has come to be popularly
associated with the distribution of content registered under certain
copyright licenses that disallow the distribution of copies (eg.
supposed criminal activity). Regardless of this new association, file
sharing remains a vital tool for many world wide: from academic groups
to scientific networks and open source software communities.

In this book we wish to help you learn to privately distribute files,
with other consenting people, without the content of that exchange known
to others or the transaction stopped by an external party. Your basic
right to anonymity and to not be spied upon protects that. Suspicions
that those things \emph{might} have been stolen and are not yours to
give does not undermine that same and original right to privacy.

The history of the internet is littered with attacks of different types
on publication and distribution nodes, conducted by different means
(court order, Distributed Denial of Service attacks). What such events
have demonstrated is that if one wants information to be persistently
available and robust against attack, it is a mistake to rely upon a
single node which can be neutralised.

This has recently been demonstrated by the takedown of the direct
download service Megaupload, whose disappearance led to the loss of
massive amounts of its users' data, much of it extraneous even to the
alleged copyright infringements which formed the pretext for its
closure. In similar vein ISPs will often take down web sites containing
disputed material merely because it is cheaper for them to do so than to
go to court and have a judge decide. Such policies leave the door open
to groundless bullying by all manner of companies, organisations and
individuals ready and willing to make aggressive use of legal letters.
Both direct download services and ISPs are examples of centralised
structures which cannot be relied upon both because they are a single
point of failure for attack, and because their commercial interests are
not aligned with those of their users.

Spreading files through distribution, decentralising the data, is the
best way to defend against such attacks. In the following section two
realms of filesharing are profiled. The first are standard p2p
technologies whose technical design is determined by the efficiency of
the networks in enabling speed of distribution and discovery of content
through associated search mechanisms. The second focuses on I2P as an
example of a so-called darknet, its design prioritises security and
anonymity over other criteria offering a robust, if less resource
efficient, path to persistent availability.

The means of sharing files mentioned below are just some examples of the
many P2P technologies that were developed since 1999. BitTorrent and
Soulseek have very different approaches, both however were designed for
easy usability by a wide public and have significant user communities.
I2P is of more recent development, has a small user base.

\textbf{BitTorrent} has become the most popular P2P file-sharing system.
The controversy that surrounds it nowadays ironically seems to help the
community grow, while police, lobbied by powerful copyright holders
seize torrent-tracker server hardware and pursue their operators,
sometimes to the point of jailing them as in the case of The Pirate Bay.

\textbf{Soulseek} - while it has never been the most popular
file-sharing platform, neither did it ever have the ambition. Soulseek
focuses on the exchange of music between enthusiasts, underground
producers, fans and researchers. The system and the community around it
is completely isolated from the Web: Soulseek files can't be linked to.
They are kept exclusively on the hard-disks of Soulseek users. The
content of the network fully depends on how many members are connected
and what they share. Files are transferred only between two users at a
time and nobody but those two users are involved. Because of this
`introverted' character - and the specificity of its content - Soulseek
has stayed out of sight of legislation and non-pro-copy copyright
advocates.

\textbf{I2P} is one of several systems developed to resist censorship
(others include FreeNet and Tor) and has a much smaller user community,
it is highlighted here because of its inclusion of Bit Torrent
functionality within its basic installation. These systems can also be
used to provide hidden services, amongst others, enabling you to publish
web pages within their environments.

\subsection{BitTorrent}

BitTorrent is a peer-to-peer (P2P) protocol that facilitates
distribution of data stored across multiple nodes/participants of the
network. There are no central servers or hubs, each node is capable of
exchanging data with any other node, sometimes hundreds of them
simultaneously. The fact that data is exchanged in parts between
numerous nodes allows for great download speeds for popular content on
BitTorrent networks, making it quickly the de facto P2P file-sharing
platform.

If you are using BitTorrent to circulate material of ambiguous legality,
you should know that enforcement agents typically collect information on
allegedly infringing peers by participating in torrent swarms, observing
and documenting the behaviour of other peers. The large number of users
creates a difficulty for the enforcement system simply at the level of
scaling up - there simply are not the resources to pursue every user.
Any court case will require actual evidence of data transfer between
your client and another (and usually evidence of you uploading), it is
enough that you provide even part of the file, not the file in its
entirety, for a prosecution to have legs. But if you prefer to lean
towards greater caution, you should use a VPN to route your BitTorrent
traffic, as detailed in the \textbf{Using VPN} chapter.

Leeching (downloading) of a file from BitTorrent network begins with a
\emph{torrent file} or \emph{magnet link}. A torrent file is a small
file containing information on the larger files you want to download.
The torrent file tells your torrent client the names of the files being
shared, a URL for the \emph{tracker} and a \emph{hash} code, which is a
unique code representing, and derived from, the underlying file - kind
of like an ID or catalog number. The client can use that hash to find
others seeding (uploading) those files, so you can download from their
computers and check the authenticity of the chunks as they arrive.

A \emph{Magnet Link} does away with the need for a torrent file and is
essentially a hyperlink containing a description for that torrent, which
your torrent client can immediately use to start finding people sharing
the file you are willing to download. Magnet links don't require a
tracker, instead they rely on \emph{Distributed Hash Table (DHT)} -
which you can read more about in the Glossary - and \emph{Peer
Exchange}. Magnet links do not refer to a file by its location (e.g.~by
IP addresses of people who have the file, or URL) but rather defines
search parameters by which this file can be found. When a magnet link is
loaded, the torrent client initiates an availability search which is
broadcast to other nodes and is basically a shout-out ``who's got
anything matching this hash?!''. Torrent client then connects to the
nodes which responded to the shout-out and begins to download the file.

BitTorrent uses encryption to prevent providers and other
man-in-the-middle from blocking and sniffing your traffic based on the
content you exchange. Since BitTorrent swarms (flocks of seeders and
leechers) are free for everyone to join it is possible for anyone to
join a swarm and gather information about all connected peers. Using
magnet links will not prevent you from being seen in a swarm; any of the
nodes sharing the same file must communicate between each-other and
thus, if just one of the nodes in your swarm is rogue, it will be able
to see your IP address. It will also be able to determine if you are
seeding the data by sending your node a download request.

One important aspect of using BitTorrent is worth a special mention.
Every chunk of data that you receive (leech) is being instantly shared
(seeded) with other BitTorrent users. Thus, a process of downloading
transforms into a process of (involuntary) publishing, using a legal
term - \emph{making available} of that data, before the download is even
complete. While BitTorrent is often used to re-distribute freely
available and legitimate software, moves, music and other materials, its
``making available'' capacity created a lot of controversy and led to
endless legal battles between copyright holders and facilitators of
BitTorrent platforms. At the moment of writing this text, the co-founder
of \emph{The Pirate Bay} Gottfrid Svartholm is being detained by Swedish
police after an international warrant was issued against him.

For these reasons, and a public relations campaign by copyright holders,
use of BitTorrent platforms has become practically analogous to piracy.
And while the meaning of terms such as piracy, copyright and ownership
in digital context is yet to be settled, many ordinary BitTorrent users
have been already prosecuted on the basis of breaking copyright laws.

Most torrent clients allow you to block IP addresses of known copyright
trolls using blacklists. Instead of using public torrents one can also
join closed trackers or use BitTorrent over VPN or Tor.

In situations when you feel that you should be worried about your
BitTorrent traffic and it's anonymity go through the following
check-list:

\begin{itemize}
\item
  Check if your torrent client supports peer-blacklists.
\item
  Check if the peer-blacklist definitions are updated on a daily basis.
\item
  Make sure your client supports all recent protocols - DHT, PEX and
  Magnet links.
\item
  Choose a torrent client that supports encrypted peers and enable it.
\item
  Upgrade or change your torrent client if any of the above mentioned
  options is not available.
\item
  Use VPN connection to disguise your BitTorrent traffic from your ISP.
  Make sure your VPN provider allows P2P traffic. See more tips and
  recommendations in Using VPN chapter.
\item
  Do not leech and seed stuff you don't know much about.
\item
  Be suspicious of high ratings and overly-positive comments regarding
  particular torrent link.
\end{itemize}
\subsection{SoulSeek}

As a peer to peer (P2P) file sharing program, the content available is
determined by the users of the Soulseek client, and what files they
choose to share. The network has historically had a diverse mix of
music, including underground and independent artists, unreleased music,
such as demos and mix-tapes, bootlegs, etc. It is is entirely financed
by donations, with no advertising or user fees.

\begin{quote}
``Soulseek does not endorse nor condone the sharing of copyrighted
materials. You should only share and download files which you are
legally allowed to, or have otherwise received permission to, share.''
(\href{http://www.soulseekqt.net}{http://www.soulseekqt.net})

\end{quote}
Soulseek network depends on a pair of central servers. One server
supports the original client and network, and the other supporting the
newer network. While these central servers are key to coordinating
searches and hosting chat rooms, they do not actually play a part in the
transfer of files between users, which takes place directly between the
users concerned.

Users can search for items; the results returned being a list of files
whose names match the search term used. Searches may be explicit or may
use wildcards/patterns or terms to be excluded. A feature specific to
the Soulseek search engine is the inclusion of the folder names and file
paths in the search list. This allows users to search by folder name.

The list of search results shows details, such as the full name and path
of the file, its size, the user who is hosting the file, together with
that users' average transfer rate, and, in the case of mp3 files, brief
details about the encoded track itself, such as bit rate, length, etc.
The resulting search list may then be sorted in a variety of ways and
individual files (or folders) chosen for download.

Unlike BitTorrent, Soulseek does not support multi-source downloading or
``swarming'' like other post-Napster clients, and must fetch a requested
file from a single source.

While the Soulseek software is free, a donation scheme exists to support
the programming effort and cost of maintaining the servers. In return
for donations, users are granted the privilege of being able to jump
ahead of non-donating users in a queue when downloading files (but only
if the files are not shared over a local area network). The Soulseek
protocol search algorithms are not published, as those algorithms run on
the server. However several Open Source implementations of server and
client software exits for Linux, OS X and Windows.

Regarding privacy and copyright issues Soulseek stand quite far away
from BitTorrent too. Soulseek has been taken to court only once, in
2008, but even that did not go anywhere. There are no indications of
Soulseek users ever being brought to court or accused of illegal
distribution of copyrighted materials or any other `digital-millenium'
crimes.

If you want to use the Soulseek network with some degree of real
anonymity, you will need to use it over a VPN.

\subsection{I2P}

I2P began as a fork from the Freenet project, originally conceived as a
mathod for censorship-resistant publishing and distribution. From their
website:

\begin{quote}
The I2P project was formed in 2003 to support the efforts of those
trying to build a more free society by offering them an uncensorable,
anonymous, and secure communication system. I2P is a development effort
producing a low latency, fully distributed, autonomous, scalable,
anonymous, resilient, and secure network. The goal is to operate
successfully in hostile environments - even when an organization with
substantial financial or political resources attacks it. All aspects of
the network are open source and available without cost, as this should
both assure the people using it that the software does what it claims,
as well as enable others to contribute and improve upon it to defeat
aggressive attempts to stifle free speech.
(\href{http://www.i2p2.de/}{http://www.i2p2.de/})

\end{quote}
For a guide to installing the software and configuring your browser see
section on Secure Filesharing - Installing I2P. Once complete, on launch
you will be brought to a console page containing links to popular sites
and services. In addition to the usual webpages (referred to as
eePsites) there are a range of applications services available ranging
from the blogging tool Syndie to a built in BitTorrent client which
functions through a web interface.
