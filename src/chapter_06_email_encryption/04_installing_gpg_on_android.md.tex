\section{Installing GPG on Android}

With the growing usage of mobile phones for e-mail, it's interesting to
be able to use GPG also on your mobile. This way you can still read the
messages sent to you in GPG on your phone and not only on your computer.

Install the \emph{Android Privacy Guard (APG)} and \emph{K--9 Mail}
applications to your Android device from the Google Play Store or
another trusted source.

\begin{enumerate}[1.]
\item
  Generate a new private key that uses DSA-Elgamal with your PC's GPG
  installation (You can only create keys with up to 1024bit key length
  on Android itself).
\item
  Copy the private key to your Android device.
\item
  Import the private key to APG. You may wish to have APG automatically
  delete the plaintext copy of your private key from your Android
  device's filesystem.
\item
  Set-up your e-mail accounts in \emph{K--9 Mail}.
\item
  In the settings for each account, under \emph{Cryptography}, make sure
  that K--9 Mail knows to use APG. You can also find options here to
  make K--9 Mail automatically sign your messages and/or encrypt them if
  APG can find a public key for the recipient(s).
\item
  Try it out.
\end{enumerate}
\subsection{APG}

This is a small tool which makes GPG encryption possible on the phone.
You can use APG to manage your private and public keys. The options in
the application are quite straightforward if you are a little knowledge
of GPG in general.

Management of keys is not very well implemented yet. The best way is to
manually copy all your public keys to the SD card in the APG folder.
Then it's easy to import your keys. After you've imported your public
and private keys, GPG encrypting, signing and decrypting will be
available for other applications as long as these applications have
integrated encryption/GPG.

\subsection{GPG enabled e-mail on Android: K--9 Mail}

The default mail application does not support GPG. Luckily there is an
excellent alternative: K--9 Mail. This application is based on the
original Android mail application but with some improvements. The
application can use APG as it's GPG provider. Setting up K--9 Mail is
straightforward and similar to setting up mail in the Android default
mail application. In the settings menu there is an option to enable
``Cryptography'' for GPG mail signing.

If you want to access your GPG mails on your phone this application is a
must have.

Please note, due to some small bugs in K--9 Mail and/or APG, it's very
advisable to disable HTML mail and use only Plain text. HTML mails are
not encrypted nicely and are often not readable.
