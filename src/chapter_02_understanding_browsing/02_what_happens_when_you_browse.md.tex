\section{What happens when you browse}

Browsing the web is communicating. You might not send as much text in
terms of number of words, but it is always the browser which initiates
and maintains the communication by requesting the bits and pieces which
are woven into what is eventually displayed on your screen.

Browsers like Mozilla Firefox, Google Chrome, Opera, Safari \& Internet
Explorer all work in a similar manner. When we type a URL (e.g.
``http://happybunnies.com'') in the address bar, the browser requests
the website (which is just a special kind of text) from a remote server
and then transforms it into colored blocks, text and images to be
displayed in the browser window. To see the text the way the browser
sees it, one just has to click on the \verb!View --> Page source! menu
entry in the browser. What comes up is the same webpage but in HTML -- a
language mainly concerned with content, context and links to other
resources (CSS and JavaScript) which govern the way these contents are
displayed and behave.

When the browser tries to open a webpage -- and assuming there are no
proxies involved -- the first thing it does is to check its own cache.
If there is no past memories of such website, it tries to resolve the
name into an address it can actually use. It is an internet program, so
it needs an Internet Protocol address (IP address or just IP). To get
this address it asks a DNS Server (kind of a telephone book for internet
programs) which is installed in the router of your internet access by
default. The IP address is a numerical label assigned to every device in
the (global) network, like the address of a house in the postal system
-- and as the address of your home, you should be very careful to whom
you hand out the IP address you are browsing from (by default this is:
to everyone). Once the IP address has been received, the browser opens a
TCP (just a communication protocol) connection to the destination host
and starts sending packages to a port at this address, typically no. 80
(ports are like doors to the servers, there are many but usually only a
few are open), unless another path is specified. These packages travel
through a number of servers on the internet (up to a couple of dozens
depending on were the target address is located). The server then looks
for the requested page and, if found, delivers it using the HTTP
protocol. (To prevent others from reading or altering the data, TLS/SSL
can be used to below HTTP to secure the connection)

When the HTTP response arrives, the browser can close the TCP connection
or reuse it for subsequent requests. The response can be one of many
things, from some sort of redirection or a classic Internal Server Error
(500). Provided the response proceeds as expected the browser will store
the page in a cache for further use, decode it (uncompress it if
compressed, rendered if video codec, etc) and display/play it according
to instructions.

Now, the process can be illustrated in a little conversation between
browser (B) and server (S):

B: ``Hallo.''

S: ``Hey!''

B: ``May I get that page with the happy bunnies, please?''

S: ``Well, here you are.''

B: ``Oh, maybe you could also give me a big version of that picture of
that bunny baby cuddling a teddy bear.''

S: ``Sure, why not.''

{[}\ldots{}{]}

B: ``That's all for now. Thank you. Bye.''

Note that there are lots of activities happening parallel to this TCP/IP
exchange. Depending on how you have configured its options, your browser
might be adding the page to browser history, saving cookies, checking
for plugins, checking for RSS updates and communicating with a variety
of servers, all while you're doing something else.

\subsection{A topography of you: footprints}

Most important: you will leave footprints. Some of them will be left on
your own computer -- a collection of cache data, browsing history and
naughty little files with elephantine memory called cookies. They are
all very convenient; speed up your browser's performance, reduce your
data download or remember your passwords and preferences from Social
Networks. They also snitch on your browsing habits and compile a record
of everywhere you go and everything you do there. This should bother you
if you are using a public computer station at a library, work at a
cybercafe, or share your apartment with a nosey partner!

Even if you configure your browser to not keep a history record, reject
cookies and delete cached files (or allocate zero MB of space for the
cache), you would still leave breadcrumbs all over the Internet. Your IP
address is recorded by default everywhere, by everyone and the packets
sent are monitored by an increasing number of entities - commercial,
governmental or criminal, along with some creeps and potential stalkers.

Democratic governments everywhere are redesigning regulations to require
Internet providers to keep a copy of everything so they can have later
access to it. In the USA, section 215 of the American PATRIOT act
\emph{`prohibits an individual or organization from revealing that it
has given records to the federal government, following an
investigation'}. That means that the company you pay every month as a
customer to provide you with Internet access can be ordered to turn over
your browsing and email records without your knowledge.

Most of the time, though, surveillance is not a 1984 affair. Google
collects your searches along with your browser identification
(\emph{user agent}), your IP and a whole bunch of data that can
eventually lead to your doorstep, but the ultimate aim is usually not
political repression but market research. Advertisers don't fuss about
advertising space any more, they just want to know everything about you.
They want to know your dietary and medication habits, how many children
you have and where you take them on holidays; how you make your money,
how much you earn and how you like to spend it. Even more: they want to
know how you \emph{feel} about stuff. They want to know if your friends
respect those feelings enough so that you can convince them to change
their consumption habits. This is not a conspiracy, but rather the
nature of Information Age capitalism. To paraphrase a famous observation
of the current situation, the best minds of our generation are thinking
about how to make people click ads.4

Some people think ads can be ignored or that having advertisers cater
for our specific needs is a win-win situation, because at least they are
spammed with things they may actually want. Even if that was the case
(it isn't): should we trust Google with such intimate details of our
life? Even if we trust Google to `do no evil', it can still be bought by
someone we do not trust; benevolent Larry Page and Sergey Brin could be
overruled by their own Board, or their data base be sequestered by a
fascistic government. One of their 30,000 employees worldwide could cut
loose and run with our data. Their servers can be hacked. And in the
end, they are just interested in their customers, \emph{the companies
paying for advertising}. We are just the product being sold.

Moreover; in the Social Networks our browsing habits are generating a
Permanent Record, a collection of data so vast that the information that
Facebook keeps about a given user alone can fill 880 pages. Nobody will
be surprised to learn that Facebook's purpose is not to make us happy --
again: if you are not paying for it, you're not the customer, you're the
product. But even if you don't care about their commercial goals,
consider this: the platform has publicly admitted hackers break into
hundreds of thousands of Facebook accounts every day.

For a taste of what lurks behind the curtains of the websites you visit,
install a plugin/add-on called \emph{Ghostery} to your browser. It's
like an x-ray-machine which reveals all the surveillance technology
which might be (and often is) embedded in a web page, normally invisible
to the user. In the same line, \emph{Do Not Track Plus} and
\emph{Trackerblock} will give you further control over online tracking,
through cookie blocking, persistent opt-out cookies, etc. Our following
chapter \textbf{Tracking} will equip you with expertise in such topics.

Even in between your computer and the router, your packages can easily
be intercepted by anyone using the same wireless network in the casual
environment of a cafe. It is a jungle out there, but still we choose
passwords like ``password'' and ``123456'', perform economic
transactions and buy tickets on public wireless networks and click on
links from unsolicited emails. It is not only our right to preserve our
privacy but also our responsibility to defend that right against the
intrusions of governments, corporations and anyone who attempts to
dispossess us. If we do not exercise those rights today, we deserve
whatever happens tomorrow.

\begin{enumerate}[1.]
\item
  If you are a Unix user, you can use the tcpdump command in the bash
  and view real time dns traffic. It's loads of fun! (and disturbing)
  \^{}
\item
  See list of TCP and UDP port numbers
  (\href{http://en.wikipedia.org/wiki/List\_of\_TCP\_and\_UDP\_port\_numbers}{http://en.wikipedia.org/wiki/List\_of\_TCP\_and\_UDP\_port\_numbers})
\item
  If this exchange is happening under an HTTPS connection, the process
  is much more complicated and also much safer, but you will find out
  more about that in a most fascinating chapter called Encryption. \^{}
\item
  This Tech Bubble Is Different
  (\href{http://www.businessweek.com/magazine/content/11\_17/b4225060960537.htm}{http://www.businessweek.com/magazine/content/11\_17/b4225060960537.htm}),
  Ashlee Vance (Businessweek magazine) \^{}
\end{enumerate}
