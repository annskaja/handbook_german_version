\section{Why Privacy Matters}

Privacy is a fundamental human right. It is recognized in many countries
to be as central to individual human dignity and social values as
Freedom of Association and Freedom of Speech. Simply put, privacy is the
border where we draw a line between how far a society can intrude into
our personal lives.

Countries differ in how they define privacy. In the UK for example,
privacy laws can be traced back to the 1300s when the English monarchy
created laws protecting people from eavesdroppers and peeping toms.
These regulations referred to the intrusion of a person's comfort and
not even the King of England could enter into a poor persons house
without their permission. From this perspective, privacy is defined in
terms of personal space and private property. In 1880 American lawyers,
Samuel Warren and Louis Brandeis described privacy as the `right to be
left alone'. In this case, privacy is synonymous with notions of
solitude and the right for a private life. In 1948, the Universal
Declaration of Human Rights specifically protected territorial and
communications privacy which by that became part of constitutions
worldwide. The European Commission on Human Rights and the European
Court of Human Rights also noted in 1978 that privacy encompasses the
right to establish relationships with others and develop emotional
well-being.

Today, a further facet of privacy increasingly perceived is the personal
data we provide to organizations, online as well as offline. How our
personal data is used and accessed drives the debate about the laws that
govern our behavior and society. This in turn has knock-on effects on
the public services we access and how businesses interact with us. It
even has effects on how we define ourselves. If privacy is about the
borders which govern who we give permission to watch us and track
aspects of our lives, then the amount and type of personal information
gathered, disseminated and processed is paramount to our basic civil
liberties.

An often heard argument, when questions of privacy and anonymity come
up, goes along the lines of, ``I only do boring stuff. Nobody will be
interested in it anyway'' or, ``I have nothing to hide''. Both of these
statements are easily defeated.

Firstly, a lot of companies are very interested in what boring things
you do precisely so they have opportunity to offer ``excellent''
products fitting interests. In this way their advertising becomes much
more efficient - they are able to tailor specifically to assumed needs
and desires. Secondly you do have lots to hide. Maybe you do not express
it in explicitly stated messages to friends and colleagues, but your
browsing - if not protected by the techniques laid out in this book -
will tell a lot about things you might rather keep secret: the
ex-partner you search for using Google, illnesses you research or movies
you watch are just few examples.

Another consideration is that just because you might not have something
to hide at this moment, you may very well in future. Putting together
all the tools and skills to protect yourself from surveillance takes
practice, trust and a bit of effort. These are things you might not be
able to achieve and configure right when you need them most and need not
take the form of a spy movie. An obsessed, persistent stalker, for
example, is enough to heavily disrupt your life. The more you follow the
suggestions given in this book, the less impact attacks like this will
have on you. Companies may also stalk you too, finding more and more
ways to reach into your daily life as the reach of computer networking
itself deepens.

Finally, a lack of anonymity and privacy does not just affect you, but
all the people around you. If a third party, like your Internet Service
Provider, reads your email, it is also violating the privacy of all the
people in your address book. This problem starts to look even more
dramatic when you look at the issues of social networking websites like
Facebook. It is increasingly common to see photos uploaded and tagged
without the knowledge or permission of the people affected.

While we encourage you to be active politically to maintain your right
to privacy, we wrote this book in order to empower people who feel that
maintaining privacy on the Internet is also a personal responsibility.
We hope these chapters will help you reach a point where you can feel
that you have some control over how much other people know about you.
Each of us has the right to a private life, a right to explore, browse
and communicate with others as one wishes, without living in fear of
prying eyes.
