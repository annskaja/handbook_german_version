\section{Secure Messaging}

SMS are short messages sent between mobile phones. The text is sent
without encryption and can be read and stored by mobile phone providers
and other parties with access to the network infrastructure to which
you're connected. To protect your messages from interception you have to
use a \emph{chat protocol} over your data connection. Thankfully this is
not at all difficult. Many Instant Messaging providers use the
\emph{Extensible Messaging and Presence Protocol (XMPP)} that allows
users to use various clients to send and receive messages and exchange
message with other providers.

Although XMPP uses TLS/SSL (see glossary entry TLS/SSL) encryption to
prevent 3rd party interception, your provider can still read your
messages and hand them over to other entities. \emph{Off-the-Record
(OTR)} Messaging however allows you encrypt your messages. The messages
you send do not have digital signatures that can be verified by a third
party, consequently the identity of their author is \emph{repudiable
afterwards}. Anyone can forge messages after a conversation to make them
look like they came from you. However, \emph{during} a conversation,
your correspondent is assured of the \emph{integrity} of the messages -
what s/he sees is authentic and unmodified.

See the section \textbf{Instant Messaging Encryption}
