\section{Tracking}

When you browse the web tiny digital traces of your presence are left
behind. Many web sites harmlessly use this data to compile statistics
and see how many people are looking at their site and which pages are
popular, but some sites go further and use various techniques to track
individual users, even going as far as trying to identify them
personally. It doesn't stop there however. Some firms store data in your
web browser which can be used to track you on other web sites. This
information can be compiled and passed on to other organizations without
your knowledge or permission.

This all sounds ominous but really who cares if some big company knows
about a few web sites that we have looked at? Big web sites compile and
use this data for ``behavioral advertising'' where ads are tailored to
fit your interests exactly. That's why after looking at say, the
Wikipedia entry for Majorca, one may suddenly start seeing lots of ads
for packaged vacations and party hats. This may seem innocent enough,
but after doing a search for ``Herpes Treatments'' or ``Fetish
Communities'' and suddenly seeing listings for relevant products, one
may start to feel that the web is getting a bit too familiar.

Such information is also of interest to other parties, like your
insurance company. If they know you have been looking at skydiving sites
or forums for congenital diseases, your premiums may mysteriously start
going up. Potential employers or landlords may turn you down based on
their concerns about your web interests. In extreme instances, the
police or tax authorities may develop an interest without you ever
having committed a crime, simply based on suspicious surfing.

\subsection{How do they track us?}

Every time you load a web page, the server software on the web site
generates a record of the page viewed in a log file. This is not always
a bad thing. When you log in to a website, there is a need for a way to
establish your identity and keep track of who you are in order to save
your preferences, or present you with customized information. It does
this by passing a small file to your browser and storing a corresponding
reference on the web server. This file is called a \emph{cookie}. It
sounds tasty but the problem is that this information stays on your
computer even after leaving the web site and may phone home to tell the
owner of the cookie about other web sites you are visiting. Some major
sites, like Facebook and Google have been caught using them to keep
track of your browsing even after you have logged out.

Supercookies / Evercookie / Zombie Cookies?

\subsection{How can I prevent tracking?}

The simplest and most direct way to deal with tracking is to delete the
cookie files in your browser:

{[}show how in Firefox (\verb!tools->Clear Recent History...!), chrome,
IE, etc. {]}

The limitation to this approach is that you will receive new cookies as
soon as you return to these sites or go to any other pages with tracking
components. The other disadvantage is that you will lose all of your
current login sessions for any open tabs, forcing you to type in
usernames and passwords again. A more convenient option, supported by
current browsers is private browsing or incognito mode. This opens a
temporary browser window that does not save the history of pages viewed,
passwords, downloaded files or cookies. Upon closing the private
browsing window, all of this information is deleted. You can enable
private browsing:

{[}show how in Firefox (\verb!tools->Start Private Browsing!), chrome,
IE, etc. {]}

This solution also has it's limitations. We cannot save bookmarks,
remember passwords, or take advantage of much of convenience offered by
modern browsers. Thankfully, there are several plugins specially
designed to address the problems of tracking. The most extensive, in
terms of features and flexibility, is Ghostery. The plugin allows you to
block categories or individual services that track users. Here's how you
install Ghostery:

{[}screenshots here installing the plugin{]}

Another option is to install an ad-blocking plugin like AdBlockPlus.
This will automatically block many of the tracking cookies sent by
advertising companies but not those used by Google, Facebook and other
web analytics companies. {[}expand on this maybe, explain ``web
analytics''{]} How can I see who is tracking me?

The easiest way to see who is tracking you is to use the Ghostery
plugin. There is a small icon on the upper right or lower right corner
of your browser window that will tell you which services are tracking
you on particular web sites.

\{Suggestion: Add Abine.com's Do Not Track add-on. I suggest using both
Ghosterly and DNT, as occasionally they block a different cookie. Abine
also has Privacy Suite, recently developed which can give a proxy
telephone and proxy email, similar to 10 Minute Mail or Guerrilla Mail
for fill- in emails for forms.\}

\subsection{A word of warning}

If you block trackers, you will have a higher level of privacy when
surfing the net. However, government agencies, bosses, hackers and
unscrupulous network administrators will still be able to intercept your
traffic and see what you are looking at. If you want to secure your
connections you will need to read the chapter on encryption. Your
identity may also be visible to other people on the internet. If you
want to thoroughly protect your identity while browsing, you will need
to take steps toward online anonymity which is explained in another
section of this book.
