\section{Secure Emails}

It is possible to send and receive secure email using standard current
email programs by adding a few add-ons. The essential function of these
add-ons is to make the message body (but not the To:, From:, CC: and
Subject: fields) unreadable by any 3rd party that intercepts or
otherwise gains access to your email or that of your conversation
partner. This process is known as encryption.

Secure email is generally done using a technique called \emph{Public-Key
Cryptography}. Public-Key Cryptography is a clever technique that uses
two code keys to send a message. Each user has a \emph{public key},
which can only be used to encrypt a message but not to decrypt it. The
public keys are quite safe to pass around without worrying that somebody
might discover them. The \emph{private keys} are kept secret by the
person who receives the message and can be used to decode the messages
that are encoded with the matching public key.

In practice, that means if Rosa wants to send Heinz a secure message,
she only needs his public key which encodes the text. Upon receiving the
email, Heinz then uses his private key to decrypt the message. If he
wants to respond, he will need to use Rosa's public key to encrypt the
response, and so on.

\subsection{What software can I use to encrypt my email?}

The most popular setup for public-key cryptography is to use \emph{Gnu
Privacy Guard (GPG)} to create and manage keys and an add-on to
integrate it with standard email software. Using GPG will give you the
option of encrypting sensitive mail and decoding incoming mail that has
been encrypted but it will not force you to use it all the time. In
years past, it was quite difficult to install and set up email
encryption but recent advances have made this process relatively simple.

See section \textbf{Email Encryption} for working with GPG in the scope
of your operating system and email program.

If you use a \emph{webmail} service and wish to encrypt your email this
is more difficult. You can use a GPG program on your computer to encrypt
the text using your public key or you can use an add-on, like Lock The
Text
(\href{http://lockthetext.sourceforge.net/}{http://lockthetext.sourceforge.net/}).
If you want to keep your messages private, we suggest using a dedicated
email program like Thunderbird instead of webmail.
