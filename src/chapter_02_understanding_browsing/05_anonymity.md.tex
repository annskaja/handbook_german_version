\section{Anonymity}

\subsection{Intro}

Article 2 of the Universal Declaration of Human Rights states:

\begin{quote}
"Everyone is entitled to all the rights and freedoms set forth in this
Declaration, without distinction of any kind, such as race, colour, sex,
language, religion, political or other opinion, national or social
origin, property, birth or other status.

\end{quote}
\begin{quote}
Furthermore, no distinction shall be made on the basis of the political,
jurisdictional or international status of the country or territory to
which a person belongs, whether it be independent, trust,
non-self-governing or under any other limitation of sovereignty.".

\end{quote}
One way of enforcing this basic right in hostile environments is by
means of anonymity, where attempts to connect an active agent to a
specific person are blocked.

Acting anonymously is also a great way to help others with a high need
for protection -- the bigger the herd of sheep, the harder it is to
target a specific one. An easy way to do so is by using TOR, a technique
which routes internet traffic between users of a special software, thus
making it untraceable to any specific IP address or person without
having control over the whole network (and nobody has that yet in the
case of the internet). A highly functional means to protect ones own
identity is by using anonymous proxy servers and Virtual Private
Networks (VPN).

\subsection{Proxy}

\begin{quote}
``An \textbf{anonymizer} or an \textbf{anonymous proxy} is a tool that
attempts to make activity on the Internet untraceable. It is a proxy
{[}server{]} computer that acts as an intermediary and privacy shield
between a client computer and the rest of the Internet. It accesses the
Internet on the user's behalf, protecting personal information by hiding
the client computer's identifying information.''
(\href{http://en.wikipedia.org/wiki/Anonymizer}{http://en.wikipedia.org/wiki/Anonymizer})

\end{quote}
The main purpose behind using a proxy is to hide or to change Internet
address (IP address) assigned to user's computer. There can be a few
reasons for needing to do so, for example:

\begin{itemize}
\item
  To anonymize access to particular server(s) and/or to obfuscate traces
  left in the log files of a web-server. For instance a user might
  need/want to access sensitive materials online (special materials,
  research topics or else) without triggering authorities attention.
\item
  To break through firewalls of corporations or repressive regimes. A
  corporation/government can limit or completely restrict Internet
  access for a particular IP address or a range of IP addresses. Hiding
  behind a proxy will help to trick these filters and access otherwise
  forbidden sites.
\item
  To watch online video and streams banned in your country due to legal
  issues.
\item
  To access websites and/or materials available only for IP addresses
  belonging to a specific country. For example, a user wants to watch a
  BBC video stream (UK-only) while not residing in the UK.
\item
  To access the Internet from a partially banned/blocked IP address.
  Public IP addresses can often have ``bad reputation'' (bandwidth
  abuse, scam or unsolicited email distribution) and be blocked by some
  web-sites and servers.
\end{itemize}
While a usual scenario would be to use proxy for accessing the Web
(HTTP), practically Internet protocol can be proxied - i.e.~sent via a
remote server. Unlike a router, proxy server is not directly forwarding
remote user requests but rather mediates those requests and echos
responses back to remote user's computer.

Proxy (unless setup as ``transparent'') does not allow direct
communication to the Internet thus applications such as browsers,
chat-clients or download applications need to be made aware of the proxy
server (see \textbf{Safer Browsing/Proxy settings} chapter)

\subsection{Tor}

\begin{quote}
\begin{itemize}
\item
  Tor prevents anyone from learning your location or browsing habits.
\item
  Tor is for web browsers, instant messaging clients, remote logins, and
  more.
\item
  Tor is free and open source for Windows, Mac, Linux/Unix, and Android.
  (\href{https://www.torproject.org}{https://www.torproject.org})
\end{itemize}
\end{quote}
Tor is a system intended to enable online anonymity, composed of client
software and a network of servers which can hide information about
users' locations and other factors which might identify them. Imagine a
message being wrapped in several layers of protection: every server
needs to take off one layer, thereby immediately deleting the sender
information of the previous server.

Use of this system makes it more difficult to trace Internet traffic to
the user, including visits to Web sites, online posts, instant messages,
and other communication forms. It is intended to protect users' personal
freedom, privacy, and ability to conduct confidential business, by
keeping their internet activities from being monitored. The software is
open-source and the network is free of charge to use.

Tor cannot and does not attempt to protect against monitoring the
traffic entering and exiting the network. While Tor does provide
protection against traffic analysis, it cannot prevent traffic
confirmation (also called end-to-end correlation). \emph{End-to-End
Correlation} is a way of matching an online identity with a real person.

A recent case of this involved the FBI wanting to prove that the man
Jeremy Hammon was behind an alias known to be responsible for several
Anonymous attacks. Sitting outside his house, the FBI were monitoring
his wireless traffic alongside a chat channel the alias was known to
visit. When Jeremy went online in his apartment, inspection of the
wireless packets revealed he was using Tor at the same moment the
suspected alias associated with him came online in the surveilled chat
channel. This was enough to incriminate Jeremy and he was arrested.

See section \textbf{Safer Browsing/Using Tor} for setup instructions.
